\chapter{Desarrollo y Construcción}

\section{Entorno de Desarrollo}
El sistema fue construido utilizando el sistema operativo \textbf{Pop!\_OS} (una distribución de Linux basada en Ubuntu), aprovechando su estabilidad y herramientas para desarrolladores. Como entorno de desarrollo integrado (IDE) se utilizó \textbf{PyCharm}, que ofrece soporte avanzado para Python y Django.

\section{Estructura del Proyecto}
El proyecto sigue la estructura estándar de una aplicación Django, organizada en "apps" o módulos para separar la funcionalidad:

\begin{verbatim}
/raiz-del-proyecto
|-- /manage.py          (Script de gestión de Django)
|-- /plan_de_estudio    (Configuración del proyecto y lógica de la aplicación)
|   |-- settings.py     (Configuración global)
|   |-- models.py       (Modelos de base de datos)
|   |-- views.py        (Controladores y lógica de negocio)
|   |-- ai_generators.py (Módulo de generación con IA)
|-- /plantillas         (Archivos HTML del Frontend)
|-- /static             (Archivos CSS, JS, imágenes de Bootstrap)
|-- /requirements.txt   (Lista de dependencias de Python)
\end{verbatim}

\section{Despliegue e Infraestructura}
El sistema se encuentra desplegado en \textbf{Railway}, una plataforma de infraestructura como servicio (PaaS). Railway facilita el despliegue continuo y la gestión de:
\begin{itemize}
    \item El servicio web de Django (Gunicorn).
    \item La base de datos PostgreSQL gestionada.
    \item Las variables de entorno para configuración segura.
\end{itemize}

\section{Funcionalidades Clave Desarrolladas}
\begin{itemize}
    \item \textbf{Formulario Dinámico:} Reemplazo de la tabla de Excel por una interfaz web amigable.
    \item \textbf{Generación con IA:} Módulo que procesa los programas de asignatura para sugerir contenido de sílabos.
    \item \textbf{Exportación:} Funcionalidad para generar archivos Excel idénticos al formato institucional original.
\end{itemize}
