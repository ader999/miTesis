\chapter{Diccionario de Datos}

A continuación se describe la estructura detallada de las principales tablas de la base de datos, correspondientes a los modelos del sistema.

\section{Entidad: Carrera}
Representa las carreras universitarias ofertadas.

\begin{table}[h]
\centering
\begin{tabular}{|l|l|p{6cm}|}
\hline
\textbf{Campo} & \textbf{Tipo} & \textbf{Descripción} \\ \hline
id & AutoField & Identificador único. \\ \hline
nombre & CharField(100) & Nombre de la carrera (Único). \\ \hline
codigo & CharField(50) & Código interno de la carrera (Único). \\ \hline
cine\_2011 & IntegerField & Clasificación CINE 2011. \\ \hline
cine\_2013 & IntegerField & Clasificación CINE 2013. \\ \hline
area\_formacion & CharField(100) & Área de formación (e.g., Educación, Tecnología). \\ \hline
area\_disiplinaria & IntegerField & Código del área disciplinaria. \\ \hline
\end{tabular}
\caption{Diccionario de datos: Tabla Carrera}
\end{table}

\section{Entidad: Asignatura}
Catálogo de asignaturas disponibles.

\begin{table}[h]
\centering
\begin{tabular}{|l|l|p{6cm}|}
\hline
\textbf{Campo} & \textbf{Tipo} & \textbf{Descripción} \\ \hline
id & AutoField & Identificador único. \\ \hline
nombre & CharField(100) & Nombre de la asignatura (Único). \\ \hline
\end{tabular}
\caption{Diccionario de datos: Tabla Asignatura}
\end{table}

\section{Entidad: Plan de Estudio}
Define la relación entre asignaturas y carreras, incluyendo detalles de carga horaria.

\begin{table}[h]
\centering
\begin{tabular}{|l|l|p{6cm}|}
\hline
\textbf{Campo} & \textbf{Tipo} & \textbf{Descripción} \\ \hline
id & AutoField & Identificador único. \\ \hline
carrera\_id & ForeignKey & Referencia a la tabla Carrera. \\ \hline
asignatura\_id & ForeignKey & Referencia a la tabla Asignatura. \\ \hline
año & CharField(4) & Año académico (Números Romanos). \\ \hline
trimestre & CharField(3) & Trimestre (Números Romanos). \\ \hline
codigo & CharField(50) & Código único del plan. \\ \hline
horas\_presenciales & IntegerField & Cantidad de horas presenciales. \\ \hline
horas\_estudio\_independiente & IntegerField & Cantidad de horas de estudio autónomo. \\ \hline
plan\_tematico & FileField & Archivo del plan temático adjunto. \\ \hline
\end{tabular}
\caption{Diccionario de datos: Tabla Plan\_de\_estudio}
\end{table}

\section{Entidad: AsignacionPlanEstudio}
Tabla intermedia que asigna un plan de estudio a un usuario (docente).

\begin{table}[h]
\centering
\begin{tabular}{|l|l|p{6cm}|}
\hline
\textbf{Campo} & \textbf{Tipo} & \textbf{Descripción} \\ \hline
id & AutoField & Identificador único. \\ \hline
usuario\_id & ForeignKey & Referencia al usuario (Docente). \\ \hline
plan\_de\_estudio\_id & ForeignKey & Referencia al Plan de Estudio. \\ \hline
fecha\_asignacion & DateTime & Fecha de creación de la asignación. \\ \hline
silabos\_creados & IntegerField & Contador de sílabos generados. \\ \hline
guias\_creadas & IntegerField & Contador de guías generadas. \\ \hline
\end{tabular}
\caption{Diccionario de datos: Tabla AsignacionPlanEstudio}
\end{table}

\section{Entidad: Silabo}
Contiene la planificación didáctica detallada de una asignatura.

\begin{table}[h]
\centering
\begin{tabular}{|l|l|p{6cm}|}
\hline
\textbf{Campo} & \textbf{Tipo} & \textbf{Descripción} \\ \hline
id & AutoField & Identificador único. \\ \hline
codigo & CharField(20) & Código del sílabo. \\ \hline
encuentros & IntegerField & Número de encuentros programados. \\ \hline
fecha & DateField & Fecha de la planificación. \\ \hline
unidad & CharField & Número de unidad (e.g., Unidad I). \\ \hline
nombre\_de\_la\_unidad & CharField & Título de la unidad temática. \\ \hline
contenido\_tematico & TextField & Descripción de los temas a tratar. \\ \hline
objetivo\_conceptual & TextField & Objetivos conceptuales (Saber). \\ \hline
objetivo\_procedimental & TextField & Objetivos procedimentales (Saber hacer). \\ \hline
objetivo\_actitudinal & TextField & Objetivos actitudinales (Saber ser). \\ \hline
asignacion\_plan\_id & ForeignKey & Referencia a la asignación del docente. \\ \hline
\end{tabular}
\caption{Diccionario de datos: Tabla Silabo (Campos principales)}
\end{table}

\section{Entidad: PlanTematico}
Define la estructura macro de las unidades de una asignatura.

\begin{table}[h]
\centering
\begin{tabular}{|l|l|p{6cm}|}
\hline
\textbf{Campo} & \textbf{Tipo} & \textbf{Descripción} \\ \hline
id & AutoField & Identificador único. \\ \hline
plan\_estudio\_id & ForeignKey & Referencia al Plan de Estudio asociado. \\ \hline
unidades & CharField & Identificador de la unidad (e.g., Primera unidad). \\ \hline
nombre\_de\_la\_unidad & CharField & Nombre descriptivo de la unidad. \\ \hline
objetivo\_especificos & TextField & Objetivos específicos de la unidad. \\ \hline
plan\_analitico & TextField & Detalle del plan analítico. \\ \hline
\end{tabular}
\caption{Diccionario de datos: Tabla PlanTematico}
\end{table}

\section{Entidad: Guia}
Representa las guías de estudio específicas para cada encuentro, vinculadas a un sílabo.

\begin{table}[h]
\centering
\begin{tabular}{|l|l|p{6cm}|}
\hline
\textbf{Campo} & \textbf{Tipo} & \textbf{Descripción} \\ \hline
id & AutoField & Identificador único. \\ \hline
silabo\_id & ForeignKey & Referencia al Sílabo padre. \\ \hline
numero\_encuentro & IntegerField & Número secuencial del encuentro. \\ \hline
fecha & DateField & Fecha de aplicación de la guía. \\ \hline
unidad & CharField & Unidad a la que pertenece. \\ \hline
objetivo\_aprendizaje\_1 & TextField & Objetivo de aprendizaje principal. \\ \hline
actividad\_aprendizaje\_1 & TextField & Descripción de la actividad. \\ \hline
tipo\_evaluacion\_1 & CharField & Tipo (Diagnóstica, Formativa, Sumativa). \\ \hline
\end{tabular}
\caption{Diccionario de datos: Tabla Guia (Campos principales)}
\end{table}
