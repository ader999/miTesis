\chapter{Guía de Instalación y Despliegue}

Esta sección detalla los pasos necesarios para instalar, configurar y desplegar el sistema \textbf{PLANEAUML} tanto en un entorno de desarrollo local como en un entorno de producción utilizando Railway.

\section{Requisitos Previos}

Antes de comenzar, asegúrese de tener instalado el siguiente software en su sistema:

\begin{itemize}
    \item \textbf{Git:} Para el control de versiones y clonado del repositorio.
    \item \textbf{Python (3.8 o superior):} Lenguaje base del backend.
    \item \textbf{Docker Desktop / Engine:} Para ejecutar el contenedor de MinIO (almacenamiento local).
    \item \textbf{PostgreSQL:} Motor de base de datos (opcional si se usa SQLite en desarrollo, pero recomendado para paridad con producción).
\end{itemize}

\section{Instalación en Entorno Local}

\subsection{1. Clonar el Repositorio}
Obtenga el código fuente del proyecto desde el repositorio oficial en GitHub:

\begin{verbatim}
git clone https://github.com/ader999/plan_de_estudio.git
cd plan_de_estudio
\end{verbatim}

\subsection{2. Configuración del Entorno Virtual}
Es recomendable utilizar un entorno virtual para aislar las dependencias del proyecto:

\begin{verbatim}
# Crear el entorno virtual
python -m venv venv

# Activar el entorno virtual
# En Windows:
venv\Scripts\activate
# En Linux/macOS:
source venv/bin/activate
\end{verbatim}

\subsection{3. Instalación de Dependencias}
Instale las librerías necesarias listadas en el archivo \texttt{requirements.txt}:

\begin{verbatim}
pip install -r requirements.txt
\end{verbatim}

\subsection{4. Configuración de Variables de Entorno}
Cree un archivo llamado \texttt{.env} en la raíz del proyecto. Este archivo debe contener las credenciales de base de datos, claves de API y configuración de almacenamiento. Copie y configure las siguientes variables:

\begin{verbatim}
# Configuración de Base de Datos
DB_HOST=localhost
DB_NAME=nombre_base_datos
DB_PASSWORD=tu_contraseña
DB_PORT=5432
DB_USER=tu_usuario

# Claves de API para IA y Servicios Externos
GEMINI_API_KEY=tu_clave_google_gemini
OPENAI_API_KEY=tu_clave_openai
RESEND=tu_clave_resend_email

# Configuración de MinIO (Almacenamiento de Archivos)
MINIO_ACCESS_KEY_ID=minioadmin
MINIO_SECRET_ACCESS_KEY=minioadmin
MINIO_BUCKET_NAME=nombre_bucket
MINIO_S3_ENDPOINT_URL=http://localhost:9000
\end{verbatim}

\subsection{5. Configuración de MinIO (Docker)}
Para el almacenamiento local de archivos, se utiliza MinIO ejecutándose en un contenedor Docker. Ejecute el siguiente comando para levantar el servicio:

\begin{verbatim}
docker run -p 9000:9000 -p 9001:9001 \
  -e "MINIO_ROOT_USER=minioadmin" \
  -e "MINIO_ROOT_PASSWORD=minioadmin" \
  quay.io/minio/minio server /data --console-address ":9001"
\end{verbatim}

Una vez activo, acceda a \texttt{http://localhost:9001}, cree un \textit{bucket} con el nombre definido en \texttt{MINIO\_BUCKET\_NAME} y asegúrese de que las credenciales coincidan con su archivo \texttt{.env}.

\subsection{6. Base de Datos y Ejecución}
Realice las migraciones para crear la estructura de la base de datos y ejecute el servidor de desarrollo:

\begin{verbatim}
# Aplicar migraciones
python manage.py migrate

# Crear superusuario (opcional, para acceso admin)
python manage.py createsuperuser

# Ejecutar servidor
python manage.py runserver
\end{verbatim}

El sistema estará accesible en \texttt{http://127.0.0.1:8000}. Asegúrese de configurar los dominios permitidos en \texttt{settings.py} si es necesario.

\section{Despliegue en Producción (Railway)}

Para desplegar el sistema en la nube utilizando Railway, siga estos pasos:

\begin{enumerate}
    \item \textbf{Inicio de Sesión:} Regístrese o inicie sesión en \href{https://railway.app/}{Railway}.
    \item \textbf{Nuevo Proyecto:} Seleccione "New Project" y elija "Deploy from GitHub repo".
    \item \textbf{Conectar Repositorio:} Busque y seleccione el repositorio \texttt{plan\_de\_estudio}.
    \item \textbf{Configuración de Variables:} Antes de finalizar el despliegue, vaya a la pestaña "Variables" del servicio y agregue todas las variables definidas en su archivo \texttt{.env} local, actualizando los valores para el entorno de producción (especialmente las credenciales de la base de datos que Railway provee automáticamente si agrega un servicio de PostgreSQL).
    \item \textbf{Despliegue:} Railway detectará automáticamente el archivo \texttt{requirements.txt} y construirá el proyecto. Una vez finalizado, le proporcionará una URL pública para acceder al sistema.
\end{enumerate}
