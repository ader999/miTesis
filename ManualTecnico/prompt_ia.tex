\section{Código Fuente: Generación de Sílabos con IA}

A continuación se presenta el fragmento de código Python encargado de construir el \textit{prompt} dinámico que se envía al modelo de Inteligencia Artificial (Gemini/OpenAI) para la generación automática de los sílabos. Este script estructura la información del plan de estudio y las reglas de negocio antes de realizar la petición a la API.

\begin{lstlisting}[language=Python, caption=Lógica de construcción del Prompt para IA]
try:
    with open(json_filepath, "r", encoding="utf-8") as json_file:
        datos_estructura = json.load(json_file)

    # Obtener la estructura del primer encuentro como ejemplo
    primer_encuentro = datos_estructura.get("primer_encuentro", {})
    # Obtener las listas de opciones
    unidades = datos_estructura.get("unidades", [])
    ejes_transversales = datos_estructura.get("ejes_transversales", [])
    tipos_primer_momento = datos_estructura.get("tipos_primer_momento", [])
    tipos_segundo_momento_teoria = datos_estructura.get(
        "tipos_segundo_momento_teoria", []
    )
    tipos_segundo_momento_practica = datos_estructura.get(
        "tipos_segundo_momento_practica", []
    )
    tipos_tercer_momento = datos_estructura.get("tipos_tercer_momento", [])

except Exception as e:
    return JsonResponse(
        {"error": f"Error al cargar el archivo de estructura JSON: {str(e)}"},
        status=400,
    )

# Crear el prompt completo
prompt_completo = f"""
    Instrucciones: Crea un sílabo basado en la siguiente información y devuélvelo en formato JSON estructurado.

    Estás creando el sílabo para el encuentro {encuentro} de 12 encuentros.
    Plan de estudio: {str(asignacion.plan_de_estudio)}
    Asignatura: {asignacion.plan_de_estudio.asignatura.nombre}

    INFORMACIÓN DEL PLAN TEMÁTICO:
    Unidad: {plan_tematico.unidades}
    Nombre de la unidad: {plan_tematico.nombre_de_la_unidad}
    Objetivos específicos: {plan_tematico.objetivo_especificos}
    Plan analítico: {plan_tematico.plan_analitico}
    Recomendaciones metodológicas: {plan_tematico.recomendaciones_metodologicas}

    CONTENIDO TEMÁTICO COMPLETO:
    {plan_tematico.plan_analitico}

    {"SÍLABOS PREVIOS YA GENERADOS:" if silabos_previos else "Este es el primer encuentro, no hay sílabos previos."}
    {json.dumps(silabos_previos, indent=2, ensure_ascii=False) if silabos_previos else ""}

    Utiliza la siguiente estructura de datos y opciones disponibles:

    UNIDADES DISPONIBLES:
    {', '.join(unidades)}

    EJES TRANSVERSALES DISPONIBLES:
    {', '.join(ejes_transversales)}

    TIPOS DE PRIMER MOMENTO DIDÁCTICO:
    {', '.join(tipos_primer_momento)}

    TIPOS DE SEGUNDO MOMENTO DIDÁCTICO (TEORÍA):
    {', '.join(tipos_segundo_momento_teoria)}

    TIPOS DE SEGUNDO MOMENTO DIDÁCTICO (PRÁCTICA):
    {', '.join(tipos_segundo_momento_practica)}

    TIPOS DE TERCER MOMENTO DIDÁCTICO:
    {', '.join(tipos_tercer_momento)}

    EJEMPLO DE ESTRUCTURA (primer encuentro):
    ```
    {json.dumps(primer_encuentro, indent=2, ensure_ascii=False)}
    ```

    INSTRUCCIONES ESPECÍFICAS:
    1. Divide el contenido temático completo en 12 encuentros de manera coherente y progresiva.
    2. Para el encuentro {encuentro}, selecciona la parte correspondiente del contenido temático.
    3. NO repitas contenido que ya se ha cubierto en encuentros anteriores.
    4. Si es el primer encuentro, comienza con una introducción general.
    5. Si hay encuentros previos, continúa desde donde se quedaron.
    6. Asegúrate de que haya una progresión lógica entre los encuentros.
    7. Adapta los objetivos y actividades al contenido específico de este encuentro.
    8. Usa la misma estructura JSON que el ejemplo proporcionado.

    Devuelve los datos como un diccionario JSON con la misma estructura que el ejemplo anterior,
    pero adaptado al encuentro {encuentro} y al plan de estudio proporcionado.

    Asegúrate de que todos los campos tengan valores coherentes y apropiados para el encuentro {encuentro}.
    Respeta las opciones disponibles para los campos que tienen listas predefinidas.
"""

try:
    # Configurar parámetros específicos para el modelo seleccionado
    config = get_default_config(modelo_seleccionado)

    # Generar respuesta usando la función centralizada
    data = generar_respuesta_ai(prompt_completo, modelo_seleccionado, **config)

    # Devolver la respuesta
    return JsonResponse({"silabo_data": data})

except Exception as e:
    error_msg = f"Error inesperado: {str(e)}"
    logging.error(error_msg)
    return JsonResponse({"error": error_msg}, status=500)
\end{lstlisting}
