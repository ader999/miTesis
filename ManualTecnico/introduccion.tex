\chapter{Introducción}

El sistema \textbf{PLANEAUML} es una aplicación web diseñada para la digitalización, optimización y gestión centralizada de los planes de estudio académicos. Su propósito principal es sustituir los procesos manuales y ofimáticos (Word/Excel) por un sistema automatizado que permita la creación de sílabos, guías de estudio y planes de clase, integrando Inteligencia Artificial para la generación de contenido y herramientas de supervisión administrativa. Este sistema fue creado específicamente para la Universidad Martín Lutero, sede Jalapa.

\section{Actores del Sistema}
Se identifican dos roles principales con permisos diferenciados:

\begin{itemize}
    \item \textbf{Administrador:} Encargado de la gestión global de datos (carreras, asignaturas, usuarios), asignación de carga académica y supervisión del progreso.
    \item \textbf{Docente (Maestro):} Usuario final encargado de completar la planificación didáctica (sílabos y guías) de las asignaturas asignadas.
\end{itemize}

Este documento está dirigido a desarrolladores, administradores de sistemas y evaluadores que deseen comprender la estructura interna y el funcionamiento técnico de la aplicación.
