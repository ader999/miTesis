
\section{Glosario Técnico}

\begin{description}
    \item[API (Interfaz de Programación de Aplicaciones):] Conjunto de definiciones y protocolos que permite que dos aplicaciones de software se comuniquen entre sí.
    \item[Backend:] Parte del desarrollo web que se encarga de la lógica del servidor, bases de datos y APIs, invisible para el usuario final.
    \item[Docker:] Plataforma de software que permite crear, probar e implementar aplicaciones rápidamente mediante contenedores, asegurando que funcionen igual en cualquier entorno.
    \item[Endpoint:] Punto final de comunicación en una API desde donde se solicitan o envían recursos específicos.
    \item[Frontend:] Parte de la aplicación web con la que interactúa el usuario directamente; comprende el diseño, la estructura y el comportamiento visual.
    \item[Framework:] Entorno de trabajo que provee una estructura base y herramientas predefinidas para facilitar y acelerar el desarrollo de software (ej. Django, Bootstrap).
    \item[LLM (Large Language Model):] Modelo de lenguaje grande entrenado con vastas cantidades de datos para entender, resumir, traducir y generar texto similar al humano (ej. GPT-4, Gemini).
    \item[MinIO:] Servidor de almacenamiento de objetos de alto rendimiento y código abierto, compatible con la API de Amazon S3, utilizado para guardar archivos de manera local o privada.
    \item[ORM (Object-Relational Mapping):] Técnica de programación que permite convertir datos entre el sistema de tipos de un lenguaje orientado a objetos (como Python) y una base de datos relacional (como PostgreSQL), facilitando la interacción con los datos.
    \item[PaaS (Platform as a Service):] Modelo de servicio en la nube que ofrece a los desarrolladores una plataforma completa para construir, desplegar y gestionar aplicaciones sin preocuparse por la infraestructura subyacente (ej. Railway).
    \item[PostgreSQL:] Sistema de gestión de bases de datos relacional de objetos, de código abierto, conocido por su fiabilidad, robustez y rendimiento.
    \item[Prompt:] Instrucción, pregunta o texto de entrada que se proporciona a una Inteligencia Artificial Generativa para guiar su respuesta o la creación de contenido.
    \item[Sprint:] En la metodología Scrum, es un periodo breve y de duración fija (usualmente de una a cuatro semanas) durante el cual el equipo trabaja para completar una cantidad específica de trabajo.
\end{description}
