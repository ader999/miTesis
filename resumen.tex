% !TEX root = main.tex


Este estudio evalúa la implementación de PLANEAUML, un sistema web para la digitalización y optimización de la planificación académica en la Universidad Martín Lutero, sede Jalapa, abordando las ineficiencias de los métodos manuales. Se empleó un Diseño Mixto Secuencial Explicativo, combinando encuestas iniciales con entrevistas posteriores. El sistema, desarrollado con Django e IA, asiste en la generación de sílabos.

Los resultados mostraron una reducción drástica en los tiempos de planificación, con el 81\% de los docentes completando la tarea en menos de una hora. Cualitativamente, se observó una adopción sostenida que transformó la cultura organizacional hacia la estandarización. El tiempo ahorrado se reinvirtió en mejorar la calidad pedagógica y la retroalimentación. La IA fue adoptada como un "asistente cognitivo" para estructurar contenidos, no como un reemplazo.

Se concluye que PLANEAUML optimizó la eficiencia administrativa y mejoró la calidad educativa. El estudio valida la digitalización centrada en el usuario como modelo replicable para la gestión en educación superior.

Palabras clave: sistema web, digitalización, planificación académica, optimización, inteligencia artificial, gestión de educación superior.
