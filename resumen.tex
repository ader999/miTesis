% !TEX root = main.tex


Este estudio evalúa la implementación de PLANEAUML, un sistema web innovador diseñado para la digitalización y optimización de la planificación académica en la Universidad Martín Lutero, sede Jalapa. La investigación aborda la problemática de los métodos tradicionales manuales, que generaban ineficiencias y sobrecarga administrativa. Se empleó un Diseño Mixto Secuencial Explicativo, combinando encuestas iniciales con entrevistas en profundidad realizadas seis meses después de la implementación. El sistema, desarrollado con Django y tecnologías modernas, integra inteligencia artificial para asistir en la generación de sílabos.

Los resultados cuantitativos iniciales mostraron una reducción drástica en los tiempos de elaboración de planes, con un 81\% de los docentes completando la tarea en menos de una hora. El seguimiento cualitativo a los seis meses reveló hallazgos profundos: la adopción del sistema se sostuvo y consolidó, transformando la cultura organizacional hacia la centralización y estandarización de la información. Se identificó que el tiempo ahorrado no se convirtió en ocio, sino que se reinvirtió en la mejora de la calidad pedagógica, permitiendo clases más dinámicas y retroalimentación personalizada. Además, la inteligencia artificial fue adoptada no como un reemplazo, sino como un "asistente cognitivo" para superar bloqueos creativos y estructurar contenidos.

Se concluye que PLANEAUML no solo optimizó la eficiencia administrativa, sino que catalizó una mejora en la calidad educativa y el bienestar docente. El estudio valida la eficacia de la digitalización centrada en el usuario y ofrece un modelo replicable para la modernización de la gestión en instituciones de educación superior.
