% !TEX root = main.tex


Este estudio evalúa la implementación de un sistema web innovador, denominado PLANEAUML, diseñado para la digitalización y optimización de los planes de estudio en la Universidad Martín Lutero, sede Jalapa, durante el primer semestre de 2025. La investigación responde a la necesidad de mejorar los procesos tradicionales de gestión académica, que resultaban manuales, propensos a errores y consumidores de tiempo valioso, limitando la eficiencia del personal docente y administrativo.
Mediante un enfoque de investigación mixta, que combina técnicas cuantitativas y cualitativas, se emplearon metodologías ágiles y tecnologías web modernas como Django, Bootstrap, PostgreSQL, y herramientas de inteligencia artificial para automatizar la generación de sílabos, minimizar errores y facilitar la gestión de la información académica. La evaluación incluyó encuestas a docentes y administrativos antes y después de la implementación, permitiendo cuantificar la reducción en los tiempos de elaboración, la satisfacción con la herramienta y la percepción de mejoras en la organización y eficiencia.

Los resultados evidenciaron que PLANEAUML logra reducir significativamente el tiempo invertido en la planificación académica, con un 81.3% de los docentes elaborando planes en menos de una hora, y una percepción general de mayor usabilidad y utilidad, tanto en el ámbito pedagógico como en el administrativo. La integración de inteligencia artificial para la generación automática de contenidos fue altamente valorada, contribuyendo a disminuir errores y optimizar recursos. Adicionalmente, la aceptación del sistema fue prácticamente unánime, fortaleciendo la gestión institucional y promoviendo una cultura digital en la institución.

Esta investigación concluye que la implementación de PLANEAUML representa una transformación positiva en la gestión académica, fortaleciendo la eficiencia, la calidad y la transparencia en los procesos académico-administrativos. Asimismo, establece un modelo replicable para otras instituciones educativas que busquen modernizar sus sistemas de planificación y gestión curricular, contribuyendo de esta forma a la innovación educativa y a la mejora continua en la educación superior.
