% !TEX root = main.tex
\label{sec:recomendaciones}

Basado en los hallazgos de esta investigación y la discusión presentada, se formulan las siguientes recomendaciones dirigidas a los diferentes actores involucrados en el proceso educativo y administrativo de la Universidad Martín Lutero.

\subsection{A la Institución (Universidad Martín Lutero)}

\begin{itemize}
    \item \textbf{Institucionalización del Sistema:} Se recomienda adoptar PLANEAUML como la herramienta oficial y estandarizada para la gestión de planes de estudio en la Sede Jalapa, eliminando gradualmente el uso de formatos dispersos en Word y Excel. Esto garantizará la centralización y seguridad de la información académica.
    \item \textbf{Escalabilidad a Otras Sedes:} Dado el éxito observado en la Sede Jalapa, se sugiere evaluar la viabilidad técnica y operativa para implementar el sistema en otras sedes de la universidad. Esto permitiría unificar criterios académicos a nivel nacional y facilitar la movilidad docente y estudiantil.
    \item \textbf{Integración de Sistemas:} A mediano plazo, se recomienda trabajar en la interoperabilidad de PLANEAUML con otros sistemas institucionales existentes, como el Sistema de Administración Universitaria (SAU) o plataformas de gestión del aprendizaje (LMS). El objetivo es crear un ecosistema digital integrado donde la información fluya automáticamente entre la planificación curricular, el registro de notas y el aula virtual.
\end{itemize}

\subsection{Para la Evolución y Mejora del Sistema}

\begin{itemize}
    \item \textbf{Mejoras Funcionales Basadas en Feedback:} Atender las solicitudes específicas de los usuarios recabadas en la evaluación, tales como:
    \begin{itemize}
        \item Unificar la "Guía de Estudio Independiente" con el "Plan de Clase" para evitar la duplicidad de información y agilizar el proceso de llenado.
        \item Implementar un sistema de notificaciones automáticas y alertas para recordar fechas de entrega y cambios en los estados de los documentos.
    \end{itemize}
    \item \textbf{Capacitación y Soporte:} Desarrollar tutoriales interactivos integrados dentro de la misma plataforma y mantener un programa de capacitación continua para los nuevos docentes que se incorporen a la institución.
\end{itemize}

\subsection{A los Docentes}

\begin{itemize}
    \item \textbf{Apropiación Tecnológica:} Se insta al cuerpo docente a continuar utilizando las herramientas de Inteligencia Artificial como asistentes cognitivos para potenciar su creatividad pedagógica, manteniendo siempre la revisión crítica de los contenidos generados.
    \item \textbf{Reinversión del Tiempo:} Se recomienda aprovechar el tiempo liberado por la automatización administrativa para fortalecer las estrategias didácticas en el aula, brindar retroalimentación más personalizada a los estudiantes y participar en actividades de investigación educativa.
\end{itemize}

\subsection{Para Futuras Investigaciones}

\begin{itemize}
    \item \textbf{Estudios Longitudinales:} Realizar un seguimiento a largo plazo para evaluar si la mejora en la planificación docente, facilitada por PLANEAUML, tiene un impacto correlacional positivo en el rendimiento académico y la satisfacción de los estudiantes.
    \item \textbf{Evaluación de Impacto en Estudiantes:} Diseñar investigaciones que incluyan la perspectiva del estudiante para medir cómo perciben ellos la organización y calidad de los sílabos y planes de clase generados con el nuevo sistema.
    \item \textbf{Análisis de Learning Analytics:} Aprovechar la base de datos estructurada que está generando el sistema para realizar estudios de analítica de aprendizaje, identificando patrones en la planificación curricular que puedan servir para la toma de decisiones pedagógicas.
\end{itemize}