% !TEX root = main.tex
\label{sec:metodologia}

\subsection{Área del Conocimiento}

El presente estudio se adscribe al Área del Conocimiento de Tecnología de la Información y la Comunicación, correspondiente a la carrera de Ingeniería en Sistemas. Sin embargo, dada la naturaleza del proyecto, la investigación se vincula estrechamente con el Área de Educación, específicamente con la carrera de Maestría en Administración de la Educación Superior con Énfasis en Currículo, ya que su objeto central es valorar un sistema web destinado a optimizar la planificación académica y la gestión de planes de estudio en la Universidad Martín Lutero. El estudio intersecta con la Gestión Educativa y Administración Académica, al evaluar la mejora en la eficiencia y reducción de carga administrativa; y con la Planificación Curricular, al considerar cómo la digitalización influye en la calidad de los sílabos. Aunque el desarrollo del sistema se fundamenta en la ingeniería de software, la valoración del mismo se enfoca en las implicaciones y beneficios de esta innovación tecnológica para los procesos de enseñanza, aprendizaje y administración educativa.

\subsection{Línea de Investigación}

De acuerdo con las líneas de investigación establecidas por la Universidad Martín Lutero, este estudio se enmarca en la línea de \textbf{Uso de las Tecnologías de la Información y Comunicación}, específicamente en la sub-línea de \textbf{Las TIC en los procesos de enseñanza y aprendizaje}. Esta clasificación responde a la naturaleza del proyecto, que consiste en la implementación de una plataforma web (TIC) para optimizar la gestión académica y apoyar la labor docente en la planificación curricular.


\subsection{Paradigma de Investigación}

La presente investigación se fundamenta en el paradigma pragmático. Este enfoque filosófico se selecciona por su naturaleza orientada a la acción y a la solución de problemas prácticos, lo cual se alinea directamente con el objetivo central del estudio: optimizar la gestión de planes de estudio mediante una herramienta tecnológica.

El pragmatismo rechaza la dicotomía estricta entre el objetivismo (propio del positivismo) y el subjetivismo (propio del constructivismo), permitiendo al investigador utilizar "lo que funciona" para responder a las preguntas de investigación. En este contexto, el paradigma justifica la adopción de un enfoque mixto, integrando datos cuantitativos (tiempos de ejecución, frecuencia de errores) para medir la eficiencia, con datos cualitativos (percepciones docentes, satisfacción) para comprender la experiencia de usuario.

Bajo esta perspectiva, el valor de la investigación no reside en descubrir una verdad universal inmutable, sino en la utilidad y aplicabilidad de la solución propuesta (PLANEAUML) dentro del contexto específico de la Universidad Martín Lutero, Sede Jalapa, buscando generar un cambio positivo y tangible en la realidad educativa.

\subsection{Enfoque de Investigación}

De conformidad con el paradigma pragmático adoptado, esta investigación se desarrolla bajo un Enfoque Mixto. Este enfoque se define como un proceso que recolecta, analiza y vincula datos cuantitativos y cualitativos en un mismo estudio para responder a un planteamiento del problema. La elección de este método responde a la complejidad del objeto de estudio: la evaluación de un sistema tecnológico en un entorno educativo, donde ni las métricas de rendimiento por sí solas, ni las opiniones aisladas, ofrecerían una visión completa del fenómeno.

\subsection{Tipo de diseño mixto}

El diseño específico seleccionado para este estudio es el de \textbf{Triangulación Concurrente}. En este modelo, los datos cuantitativos y cualitativos se recolectan de manera simultánea y se analizan para comparar y contrastar los resultados. La meta es validar los hallazgos de un método con los del otro, logrando una comprensión más robusta del fenómeno estudiado.

La aplicación de este diseño en la investigación se estructura de la siguiente manera:

\begin{itemize}
    \item \textbf{Componente Cuantitativo:} Se orienta a medir con precisión la eficiencia operativa del sistema PLANEAUML. Busca responder a preguntas sobre "cuánto" y "con qué frecuencia", proporcionando datos objetivos sobre la reducción de tiempos en la elaboración de sílabos y la disminución de errores técnicos.
    
    \item \textbf{Componente Cualitativo:} Se enfoca en comprender la experiencia subjetiva de los usuarios. Busca profundizar en el "por qué" y el "cómo", explorando la satisfacción de los docentes, la facilidad de uso percibida, la integración de la herramienta en su rutina diaria y el impacto pedagógico que estos perciben tras la implementación.
\end{itemize}

La integración de ambas perspectivas permite una validación cruzada, asegurando que las mejoras estadísticas en eficiencia se correspondan con una percepción real de bienestar y utilidad por parte de los docentes.

\subsection{Perspectiva Cuantitativa}

La perspectiva cuantitativa en este estudio se orienta a la recolección y análisis de datos objetivos y medibles, implementada en dos fases clave de la investigación. En una primera instancia, se aplicó un instrumento diagnóstico para identificar y dimensionar las deficiencias en los métodos tradicionales de gestión de planes de estudio. Posteriormente, tras la implementación del sistema, se realizó una segunda medición para evaluar el impacto de la solución tecnológica PLANEAUML. Su propósito principal es cuantificar la eficiencia operativa, la reducción de tiempos y el nivel de usabilidad, permitiendo así validar las hipótesis planteadas y determinar la magnitud de la mejora lograda.

\subsubsection{Tipo de Investigación}

De acuerdo con el nivel de profundidad y los objetivos planteados, la investigación se clasifica como Descriptiva y Explicativa.

Es Descriptiva porque se detallan las características del fenómeno estudiado en su estado inicial, especificando las propiedades de los procesos de planificación académica y midiendo variables como los tiempos de ejecución y la frecuencia de errores antes de la intervención. Esta fase permitió establecer una línea base clara sobre la problemática existente.

Es Explicativa porque no se limita a describir el problema, sino que busca determinar el efecto causal que produce la implementación del sistema PLANEAUML sobre las variables dependientes (eficiencia y satisfacción). El estudio pretende explicar cómo y por qué la introducción de esta tecnología y sus componentes de inteligencia artificial modifican el comportamiento de los indicadores de gestión académica.

\subsubsection{Población y Muestra}


La población de interés para el estudio cuantitativo se definió como el conjunto de 32 docentes activos de la Universidad Martín Lutero, sede Jalapa, durante el cuarto trimestre del año 2024. Este grupo constituye el universo total de individuos directamente afectados por los procesos de planificación académica que el sistema PLANEAUML busca optimizar.


Para la recolección de datos se empleó un muestreo no probabilístico por conveniencia. El procedimiento se llevó a cabo en dos etapas distintas, correspondiendo a cada una de las encuestas aplicadas:


\textbf{Muestra para la Encuesta de Diagnóstico (Pre-intervención):} Se distribuyó un enlace al formulario de Google Forms en el grupo de WhatsApp que reúne a todos los docentes activos de la universidad. De los 32 miembros de la población, se obtuvo una muestra final de 16 docentes que respondieron voluntariamente. Esta muestra, que representa el 50\% de la población docente, se considera suficiente para cumplir los objetivos descriptivos de la primera fase de la investigación dentro del contexto específico de la institución.
    
\textbf{Muestra para la Encuesta de Evaluación (Post-intervención):} Para esta segunda encuesta, la muestra fue seleccionada intencionalmente, incluyendo únicamente a los 16 docentes que habían utilizado previamente el sistema PLANEAUML. Este criterio aseguró que los datos cuantitativos sobre la valoración del sistema provinieran exclusivamente de usuarios con experiencia directa.


\subsubsection{Técnicas e Instrumentos de Recolección de Datos}

Para la recolección de datos cuantitativos, la técnica principal fue la encuesta, implementada a través de un cuestionario estructurado como instrumento de medición. Este enfoque se seleccionó por su capacidad para recopilar datos numéricos y categóricos de manera estandarizada, lo cual es esencial para realizar un análisis estadístico riguroso y comparativo. La administración de los cuestionarios se llevó a cabo mediante la plataforma Google Forms, garantizando una distribución ágil, una recolección centralizada y un almacenamiento eficiente de las respuestas.

Se diseñaron dos instrumentos distintos para capturar la evolución del proceso. El primer cuestionario, titulado "Evaluación del Proceso de Planeación Académica", se aplicó antes de la intervención con el objetivo de establecer una línea base cuantitativa del método tradicional. Estaba compuesto por preguntas cerradas con opciones predefinidas para medir variables clave como el tiempo promedio de elaboración, la frecuencia de errores y el nivel de satisfacción con las herramientas existentes. Posteriormente, se aplicó un segundo cuestionario, "Evaluación del sistema PLANEAUML", tras la implementación del sistema. Su finalidad era medir el impacto y la percepción de la nueva herramienta a través de preguntas con escalas de valoración tipo Likert.

\subsubsection{Operacionalización de Variables}

A continuación, se presenta la operacionalización de las variables del estudio, detallando su definición conceptual e indicadores de medición:

\begin{longtable}{|p{0.25\linewidth}|p{0.35\linewidth}|p{0.3\linewidth}|}
\caption{Operacionalización de Variables} \label{tab:operacionalizacion} \\
\hline
\textbf{Variable} & \textbf{Definición Conceptual} & \textbf{Indicadores} \\
\hline
\multicolumn{3}{|c|}{\textbf{Variable Independiente}} \\
\hline
Implementación del sistema web PLANEAUML & Intervención tecnológica consistente en la incorporación y uso de la plataforma web desarrollada, que integra módulos de gestión académica e IA generativa en los procesos de planificación. & - Uso de módulos de gestión.\newline - Uso de herramientas de IA. \\
\hline
\multicolumn{3}{|c|}{\textbf{Variables Dependientes}} \\
\hline
Optimización de la Gestión de Planes de Estudio & Mejora en la eficiencia operativa y la calidad del proceso de planificación académica. & - Tiempo promedio de elaboración de sílabos (horas).\newline - Frecuencia de errores o retrabajos.\newline - Cumplimiento de plazos. \\
\hline
Satisfacción del Usuario & Grado de aceptación, bienestar y valoración positiva del sistema por parte del personal docente y administrativo. & - Percepción de facilidad de uso.\newline - Utilidad percibida (especialmente IA).\newline - Preferencia frente a métodos tradicionales. \\
\hline
\multicolumn{3}{|c|}{\textbf{Variable de Control}} \\
\hline
Competencias Digitales Previas & Nivel de habilidades básicas en el uso de herramientas informáticas e internet que poseen los docentes antes de la intervención. & - Manejo de herramientas ofimáticas.\newline - Navegación en entornos web. \\
\hline
\end{longtable}

