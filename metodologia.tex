% !TEX root = main.tex
\label{sec:metodologia}

\subsection{Área del Conocimiento}

El presente estudio se adscribe al Área del Conocimiento de Tecnología de la Información y la Comunicación, correspondiente a la carrera de Ingeniería en Sistemas. Sin embargo, dada la naturaleza del proyecto, la investigación se vincula estrechamente con el Área de Educación, específicamente con la carrera de Maestría en Administración de la Educación Superior con Énfasis en Currículo, ya que su objeto central es valorar un sistema web destinado a optimizar la planificación académica y la gestión de planes de estudio en la Universidad Martín Lutero. El estudio converge con la Gestión Educativa y Administración Académica, al evaluar la mejora en la eficiencia y reducción de carga administrativa; y con la Planificación Curricular, al considerar cómo la digitalización influye en la calidad de los sílabos. Aunque el desarrollo del sistema se fundamenta en la ingeniería de software, la valoración del mismo se enfoca en las implicaciones y beneficios de esta innovación tecnológica para los procesos de enseñanza, aprendizaje y administración educativa.

\subsection{Línea de Investigación}

De acuerdo con las líneas de investigación establecidas por la Universidad Martín Lutero, este estudio se enmarca en la línea de \textbf{Uso de las Tecnologías de la Información y Comunicación}, específicamente en la sub-línea de \textbf{Las TIC en los procesos de enseñanza y aprendizaje}. Esta clasificación responde a la naturaleza del proyecto, que consiste en la implementación de una plataforma web (TIC) para optimizar la gestión académica y apoyar la labor docente en la planificación curricular.


\subsection{Paradigma de Investigación}

La presente investigación se fundamenta en el paradigma pragmático \cite{creswell2014}. Este enfoque filosófico se selecciona por su naturaleza orientada a la acción y a la solución de problemas prácticos, lo cual se alinea directamente con el objetivo central del estudio: optimizar la gestión de planes de estudio mediante una herramienta tecnológica.

El pragmatismo rechaza la dicotomía estricta entre el objetivismo (propio del positivismo) y el subjetivismo (propio del constructivismo), permitiendo al investigador utilizar "lo que funciona" para responder a las preguntas de investigación. En este contexto, el paradigma justifica la adopción de un enfoque mixto, integrando datos cuantitativos (tiempos de ejecución, frecuencia de errores) para medir la eficiencia, con datos cualitativos (percepciones docentes, satisfacción) para comprender la experiencia de usuario.

Bajo esta perspectiva, el valor de la investigación no reside en descubrir una verdad universal inmutable, sino en la utilidad y aplicabilidad de la solución propuesta (PLANEAUML) dentro del contexto específico de la Universidad Martín Lutero, Sede Jalapa, buscando generar un cambio positivo y tangible en la realidad educativa.

\subsection{Enfoque de Investigación}

De conformidad con el paradigma pragmático adoptado, esta investigación se desarrolla bajo un Enfoque Mixto \cite{hernandez2014}. Este enfoque se define como un proceso que recolecta, analiza y vincula datos cuantitativos y cualitativos en un mismo estudio para responder a un planteamiento del problema. La elección de este método responde a la complejidad del objeto de estudio: la evaluación de un sistema tecnológico en un entorno educativo, donde ni las métricas de rendimiento por sí solas, ni las opiniones aisladas, ofrecerían una visión completa del fenómeno.

\subsection{Tipo de diseño mixto}

El diseño específico seleccionado para este estudio es el de \textbf{Diseño Mixto Secuencial Explicativo (DEXPLIS)} \cite{creswell2018}. En este modelo, la recolección y análisis de datos se realiza en dos etapas consecutivas: primero se recaban los datos cuantitativos y posteriormente los cualitativos. La meta es utilizar los resultados cualitativos para explicar, profundizar y comprender los hallazgos cuantitativos iniciales.

La aplicación de este diseño en la investigación se estructura de la siguiente manera:

\begin{itemize}
    \item \textbf{Fase 1 (Cuantitativa):} Se orienta a medir con precisión la eficiencia operativa del sistema PLANEAUML. Busca responder a preguntas sobre "cuánto" y "con qué frecuencia", proporcionando datos objetivos sobre la reducción de tiempos en la elaboración de sílabos y la disminución de errores técnicos.
    
    \item \textbf{Fase 2 (Cualitativa):} Se enfoca en comprender la experiencia subjetiva de los usuarios. Busca profundizar en el "por qué" y el "cómo", explorando la satisfacción de los docentes, la facilidad de uso percibida, la integración de la herramienta en su rutina diaria y el impacto pedagógico que estos perciben tras la implementación.
\end{itemize}

La integración de ambas perspectivas permite una explicación robusta, asegurando que las mejoras estadísticas en eficiencia se correspondan con una percepción real de bienestar y utilidad por parte de los docentes.

\subsection{Perspectiva Cuantitativa}

La perspectiva cuantitativa en este estudio se orienta a la recolección y análisis de datos objetivos y medibles, implementada en dos fases clave de la investigación. En una primera instancia, se aplicó un instrumento diagnóstico para identificar y dimensionar las deficiencias en los métodos tradicionales de gestión de planes de estudio. Posteriormente, tras la implementación del sistema, se realizó una segunda medición para evaluar el impacto de la solución tecnológica PLANEAUML. Su propósito principal es cuantificar la eficiencia operativa, la reducción de tiempos y el nivel de usabilidad, permitiendo así validar las hipótesis planteadas y determinar la magnitud de la mejora lograda.

\subsubsection{Tipo de Investigación}

De acuerdo con el nivel de profundidad y los objetivos planteados, la investigación se clasifica como Descriptiva y Explicativa.

Es Descriptiva porque se detallan las características del fenómeno estudiado en su estado inicial, especificando las propiedades de los procesos de planificación académica y midiendo variables como los tiempos de ejecución y la frecuencia de errores antes de la intervención. Esta fase permitió establecer una línea base clara sobre la problemática existente.

Es Explicativa porque no se limita a describir el problema, sino que busca determinar el efecto causal que produce la implementación del sistema PLANEAUML sobre las variables dependientes (eficiencia y satisfacción). El estudio pretende explicar cómo y por qué la introducción de esta tecnología y sus componentes de inteligencia artificial modifican el comportamiento de los indicadores de gestión académica.

\subsubsection{Población y Muestra}


La población de interés para el estudio se definió como el conjunto de \textbf{32 docentes activos y el personal administrativo} de la Universidad Martín Lutero, sede Jalapa, durante el cuarto trimestre del año 2024. Este grupo constituye el universo total de individuos directamente afectados por los procesos de planificación académica que el sistema PLANEAUML busca optimizar.


Para la recolección de datos se empleó un muestreo no probabilístico por conveniencia. El procedimiento se llevó a cabo en dos etapas distintas, correspondiendo a cada una de las encuestas aplicadas:


\paragraph{Muestra para la Encuesta de Diagnóstico y Evaluación (Fases 1 y 2)} Para las primeras etapas de la investigación, correspondientes al diagnóstico y la evaluación inmediata tras la implementación, se contó con una muestra de 16 docentes. Esta selección, realizada mediante un muestreo no probabilístico por conveniencia, incluyó a los profesores que estuvieron activos y disponibles durante el periodo inicial de despliegue del sistema. Este grupo representa el 50\% de la población total y proporcionó los datos cuantitativos base para medir la eficiencia y usabilidad inicial de PLANEAUML.


\subsubsection{Técnicas e Instrumentos de Recolección de Datos}

Para la recolección de datos cuantitativos, la técnica principal fue la encuesta, implementada a través de un cuestionario estructurado como instrumento de medición. Este enfoque se seleccionó por su capacidad para recopilar datos numéricos y categóricos de manera estandarizada, lo cual es esencial para realizar un análisis estadístico riguroso y comparativo. La administración de los cuestionarios se llevó a cabo mediante la plataforma Google Forms, garantizando una distribución ágil, una recolección centralizada y un almacenamiento eficiente de las respuestas.

Se diseñaron dos instrumentos distintos para capturar la evolución del proceso. El primer cuestionario, titulado "Evaluación del Proceso de Planeación Académica", se aplicó antes de la intervención con el objetivo de establecer una línea base cuantitativa del método tradicional. Estaba compuesto por preguntas cerradas con opciones predefinidas para medir variables clave como el tiempo promedio de elaboración, la frecuencia de errores y el nivel de satisfacción con las herramientas existentes. Posteriormente, se aplicó un segundo cuestionario, "Evaluación del sistema PLANEAUML", tras la implementación del sistema. Su finalidad era medir el impacto y la percepción de la nueva herramienta a través de preguntas con escalas de valoración tipo Likert.

\subsubsection{Operacionalización de Variables}

A continuación, se presenta la operacionalización de las variables del estudio, detallando su definición conceptual e indicadores de medición:

\begin{longtable}{|p{0.25\linewidth}|p{0.35\linewidth}|p{0.3\linewidth}|}
\caption{Operacionalización de Variables} \label{tab:operacionalizacion} \\
\hline
\textbf{Variable} & \textbf{Definición Conceptual} & \textbf{Indicadores} \\
\hline
\multicolumn{3}{|c|}{\textbf{Variable Independiente}} \\
\hline
Implementación del sistema web PLANEAUML & Intervención tecnológica consistente en la incorporación y uso de la plataforma web desarrollada, que integra módulos de gestión académica e IA generativa en los procesos de planificación. & - Uso de módulos de gestión.\newline - Uso de herramientas de IA. \\
\hline
\multicolumn{3}{|c|}{\textbf{Variables Dependientes}} \\
\hline
Optimización de la Gestión de Planes de Estudio & Mejora en la eficiencia operativa y la calidad del proceso de planificación académica. & - Tiempo promedio de elaboración de sílabos (horas).\newline - Frecuencia de errores o retrabajos.\newline - Cumplimiento de plazos. \\
\hline
Satisfacción del Usuario & Grado de aceptación, bienestar y valoración positiva del sistema por parte del personal docente y administrativo. & - Percepción de facilidad de uso.\newline - Utilidad percibida (especialmente IA).\newline - Preferencia frente a métodos tradicionales. \\
\hline
\multicolumn{3}{|c|}{\textbf{Variable de Control}} \\
\hline
Competencias Digitales Previas & Nivel de habilidades básicas en el uso de herramientas informáticas e internet que poseen los docentes antes de la intervención. & - Manejo de herramientas ofimáticas.\newline - Navegación en entornos web. \\
\hline
\end{longtable}

\subsubsection{Confiabilidad y Validez de los Instrumentos}

Para garantizar el rigor científico de la investigación, los instrumentos de recolección de datos fueron sometidos a procesos de validación y análisis de confiabilidad.

\paragraph{Validez} La validez de contenido de los cuestionarios "Evaluación del Proceso de Planeación Académica" y "Evaluación del sistema PLANEAUML" se determinó mediante el juicio de expertos. Un panel de especialistas en las áreas de gestión educativa, currículo y tecnología evaluó la pertinencia, claridad y coherencia de los ítems con respecto a los objetivos del estudio y las variables operacionalizadas. Las observaciones y sugerencias de los expertos fueron incorporadas para refinar los instrumentos antes de su aplicación final, asegurando que estos midieran efectivamente las dimensiones de eficiencia operativa y satisfacción del usuario.

\paragraph{Confiabilidad} La confiabilidad de los instrumentos, entendida como la consistencia interna de las mediciones, se evaluó utilizando el coeficiente Alfa de Cronbach, particularmente para las secciones de los cuestionarios que emplearon escalas tipo Likert (como la medición de satisfacción y usabilidad en la fase post-intervención). Este análisis estadístico permitió verificar que los ítems de cada dimensión presentaran una correlación adecuada entre sí, garantizando que los resultados obtenidos fuesen precisos y estables. Para las variables objetivas de la fase diagnóstica (tiempos y frecuencia de errores), la confiabilidad se sustentó en la estandarización de las preguntas y las opciones de respuesta cerradas.

\subsubsection{Procesamiento y Análisis de Datos}

Una vez recolectados los datos a través de Google Forms, estos fueron exportados a una hoja de cálculo (Google Sheets) para su procesamiento y análisis. El análisis de los datos cuantitativos se realizó mediante estadística descriptiva.

El procedimiento consistió en:

\begin{itemize}
    \item \textbf{Depuración de datos:} Se revisó la base de datos para asegurar la consistencia e integridad de las respuestas.
    \item \textbf{Tabulación:} Se organizaron las respuestas en tablas de frecuencia para cada una de las preguntas de los cuestionarios.
    \item \textbf{Cálculo de estadísticos:} Se calcularon principalmente frecuencias y porcentajes para describir las respuestas de los participantes. Estos resultados se utilizaron para crear tablas y gráficos (de barras y circulares) que permitieran una visualización clara de los hallazgos y facilitaran la comparación entre el estado previo y posterior a la implementación de PLANEAUML. El software utilizado para la generación de gráficos y el análisis fue Google Sheets.
\end{itemize}

\subsection{Perspectiva Cualitativa}

La perspectiva cualitativa de esta investigación se centra en comprender la profundidad de la experiencia humana frente a la innovación tecnológica. Mientras que los datos cuantitativos nos indican la magnitud de la mejora en términos de eficiencia, el enfoque cualitativo busca explorar el "por qué" y el "cómo" de estos cambios desde la voz de los protagonistas. Esta fase del estudio se orienta a capturar las percepciones, significados y valoraciones subjetivas de los docentes y administrativos respecto a la implementación de PLANEAUML. Se indaga en cómo la herramienta se integra en sus rutinas diarias, cómo modifica su carga cognitiva y emocional frente al trabajo administrativo, y de qué manera perciben que la inteligencia artificial influye en su rol pedagógico. El objetivo es trascender las métricas de rendimiento para revelar el impacto cultural y vivencial de la transformación digital en la institución.

\subsubsection{Enfoque Cualitativo Asumido y su Justificación}

Para el abordaje de la dimensión cualitativa, se asumió un \textbf{enfoque fenomenológico-hermenéutico}. Este enfoque es el más pertinente para esta investigación ya que no busca simplemente medir variables, sino comprender la estructura de la experiencia vivida por los docentes al transitar de un sistema manual a uno automatizado. La fenomenología permite describir fielmente cómo los usuarios perciben y experimentan el cambio tecnológico en su "mundo de la vida" laboral, mientras que la hermenéutica facilita la interpretación de los significados que estos otorgan a dicha transformación.

La justificación de este enfoque radica en la necesidad de complementar la rigidez del dato numérico con la riqueza del contexto humano. Específicamente:

\begin{itemize}
    \item \textbf{Profundidad sobre la Eficiencia:} Mientras lo cuantitativo indica una reducción de horas de trabajo, el enfoque cualitativo revela la \textit{cualidad} de ese tiempo liberado. Permite entender que la eficiencia no es un fin en sí mismo, sino un medio para la "reinversión pedagógica", donde el tiempo ahorrado se transforma en mejor atención al estudiante y creatividad didáctica.
    
    \item \textbf{Interacción Humano-IA:} La adopción de inteligencia artificial genera incertidumbres y expectativas que una encuesta cerrada no logra capturar. Este enfoque permitió desentrañar la relación real entre el docente y el algoritmo, pasando del temor al reemplazo a la valoración de la IA como un "asistente cognitivo" que supera el bloqueo del papel en blanco.
    
    \item \textbf{Dimensión Emocional del Cambio:} La transición digital conlleva un cambio en el estado anímico del personal. El análisis cualitativo justificó su relevancia al evidenciar el paso de la frustración y el agobio (causados por la dispersión de archivos) a una sensación de control, seguridad y coherencia institucional.
\end{itemize}

En síntesis, este enfoque se justifica porque la implementación de PLANEAUML no es solo un evento técnico, sino un fenómeno social y educativo complejo que requiere ser interpretado desde la subjetividad de quienes lo protagonizan.

\subsubsection{Sujetos del Estudio y Muestra Teórica}

En coherencia con el paradigma cualitativo, la selección de los participantes no respondió a criterios de representatividad estadística, sino a la pertinencia para comprender el fenómeno de estudio. Los sujetos del estudio fueron los docentes y personal administrativo de la Universidad Martín Lutero, Sede Jalapa, que participaron directamente en la implementación del sistema PLANEAUML.

Se utilizó un muestreo de tipo \textbf{teórico o intencional}. Bajo este criterio, se seleccionó a los participantes que poseían la experiencia vivencial necesaria para aportar información rica y profunda sobre la transición tecnológica.
    
Es importante destacar que esta fase cualitativa se llevó a cabo \textbf{seis meses después de la implementación inicial} del sistema. Esta temporalidad permitió capturar no solo la reacción inmediata, sino la adopción sostenida de la herramienta. Debido a este lapso, el número de usuarios activos con experiencia significativa aumentó, permitiendo contar con una muestra final de \textbf{18 docentes} (dos más que en la fase cuantitativa inicial).
    
Los criterios de inclusión fueron:
\begin{itemize}
    \item \textbf{Experiencia Directa y Sostenida:} Docentes que habían utilizado PLANEAUML durante al menos un ciclo académico completo (6 meses), garantizando una vivencia profunda del cambio.
    \item \textbf{Contraste Histórico:} Tener experiencia previa con los métodos tradicionales (Word/Excel) en la misma institución.
    \item \textbf{Voluntariedad:} Disposición expresa para compartir sus percepciones a través de entrevistas y encuestas de seguimiento.
\end{itemize}

El tamaño de la muestra (18) obedeció al criterio de \textit{saturación de categorías}, donde la recolección de información cesó cuando los relatos comenzaron a mostrar patrones recurrentes y no emergían nuevos hallazgos significativos.

\subsubsection{Métodos y Técnicas de Recolección de datos}

En correspondencia con la naturaleza cualitativa del estudio y el enfoque fenomenológico-hermenéutico adoptado, se seleccionaron técnicas de recolección de datos orientadas a capturar la subjetividad, los significados y las experiencias vividas por los participantes. El objetivo no fue la estandarización, sino la profundidad y la comprensión del fenómeno desde la perspectiva interna de los actores.

Para ello, se emplearon las siguientes técnicas e instrumentos:

\paragraph{Entrevista Semiestructurada en Profundidad} Se constituyó como la técnica principal de indagación cualitativa. Se optó por la modalidad semiestructurada debido a su flexibilidad, lo que permitió al investigador contar con una guía de temas clave (guion de entrevista) pero con la libertad de adaptar el flujo de la conversación para profundizar en áreas emergentes o inesperadas que resultaran significativas para el entrevistado.

El instrumento utilizado fue una \textit{Guía de Entrevista}, diseñada para explorar dimensiones como: la carga emocional y laboral de los métodos tradicionales, el proceso de adaptación a la nueva tecnología, la percepción de utilidad de la inteligencia artificial y los cambios experimentados en su práctica docente. Estas entrevistas se realizaron fomentando un clima de confianza que facilitó la narración honesta de sus vivencias.

\paragraph{Preguntas Abiertas (Cuestionario Cualitativo)} Como complemento a las entrevistas y en coherencia con el diseño mixto, se incorporaron preguntas abiertas dentro de los instrumentos de encuesta. Esta técnica permitió recolectar narrativas breves y opiniones de un grupo más amplio de participantes, quienes pudieron justificar sus respuestas cerradas y describir con sus propias palabras los beneficios o dificultades encontradas.
    
\textbf{Encuesta de Seguimiento Longitudinal (Fase 3)}

Se aplicó un tercer instrumento seis meses después de la implementación. Este cuestionario mixto (preguntas cerradas y abiertas) tuvo como objetivo evaluar la sostenibilidad de los cambios, la integración profunda del sistema en la cultura institucional y el impacto a mediano plazo en la calidad pedagógica. Esta fase fue crucial para validar si los beneficios iniciales se mantenían o evolucionaban con el tiempo.



\subsubsection{Métodos y Técnicas para el Análisis de Información}

Para el procesamiento y análisis de los datos cualitativos, se siguió un proceso sistemático y riguroso orientado a la reducción de datos, la disposición y transformación de la información, y la obtención de resultados y verificación de conclusiones. Dado el enfoque fenomenológico-hermenéutico, el análisis no se limitó a la descripción superficial, sino que buscó interpretar los significados subyacentes en las narrativas de los docentes.

El procedimiento analítico se estructuró en las siguientes fases:

\paragraph{1. Transcripción y Organización de Datos} Las entrevistas semiestructuradas fueron grabadas y posteriormente transcritas en su totalidad para asegurar la fidelidad de los testimonios. De igual manera, las respuestas a las preguntas abiertas de los cuestionarios se recopilaron y organizaron en matrices de texto. Se realizó una lectura flotante inicial de todo el material para obtener una visión general y familiarizarse con el contenido antes de proceder a la codificación.

\paragraph{2. Codificación y Categorización} Se empleó la técnica de \textbf{Análisis de Contenido Temático}. Este proceso consistió en identificar unidades de significado (frases o párrafos relevantes) y asignarles códigos descriptivos. Posteriormente, estos códigos se agruparon en categorías emergentes que representaban los temas centrales de la experiencia de los usuarios. Las categorías principales identificadas incluyeron: "Transformación de la Rutina", "Impacto en la Calidad Educativa", "Interacción con la IA" y "Adopción Tecnológica". Este proceso fue inductivo, permitiendo que las categorías surgieran de los propios datos en lugar de imponer estructuras predefinidas.

\paragraph{3. Integración de Datos} En coherencia con el diseño mixto secuencial, se integraron los hallazgos obtenidos de las diferentes etapas. Se utilizaron las narrativas de las entrevistas en profundidad para explicar y contextualizar los resultados estadísticos cuantitativos. Esta integración permitió enriquecer la interpretación, asegurando que las tendencias numéricas (ej. reducción de tiempo) estuvieran respaldadas por explicaciones vivenciales (ej. sensación de alivio y reinversión pedagógica).

\paragraph{4. Interpretación Hermenéutica} Finalmente, se procedió a la interpretación profunda de las categorías, buscando comprender el sentido que los docentes otorgan a la transición digital. Se analizó cómo el cambio de herramienta modificó no solo sus procesos operativos, sino también su percepción de competencia profesional y su relación con la tecnología en el aula.

Este proceso analítico garantizó que los resultados presentados reflejaran fielmente la realidad subjetiva de los participantes, cumpliendo con los criterios de rigor metodológico propios de la investigación cualitativa.

\subsubsection{Criterios de Calidad}

Para asegurar la rigurosidad científica y la confiabilidad de los hallazgos cualitativos, este estudio se adhirió a los criterios de calidad propuestos por Guba y Lincoln (1985), quienes establecen la credibilidad, transferibilidad, dependencia y confirmabilidad como los estándares para evaluar la confiabilidad en la investigación naturalista.

\paragraph{Credibilidad} Este criterio, análogo a la validez interna, se logró principalmente a través de la \textbf{triangulación de fuentes}. Se contrastaron las perspectivas obtenidas de las entrevistas en profundidad con las respuestas de los cuestionarios abiertos y los datos cuantitativos, buscando convergencia y consistencia en los hallazgos. Además, se alcanzó la \textbf{saturación de categorías}, asegurando que la recolección de datos continuara hasta que no emergiera nueva información significativa, garantizando así una comprensión completa del fenómeno.

\paragraph{Transferibilidad} Aunque la investigación cualitativa no busca la generalización estadística, se facilitó la transferibilidad (o aplicabilidad) mediante una \textbf{descripción densa} del contexto y de los participantes. Se han detallado las características de la sede universitaria, el perfil de los docentes y los procedimientos de gestión académica, permitiendo que otros investigadores evalúen la posibilidad de transferir estos hallazgos a contextos educativos similares que enfrenten procesos de transformación digital.

\paragraph{Dependencia y Confirmabilidad} Para garantizar la consistencia (dependencia) y la neutralidad (confirmabilidad) del estudio, se mantuvo un registro sistemático de todo el proceso de investigación. Se documentaron las decisiones metodológicas, las grabaciones de las entrevistas y las transcripciones textuales, creando una \textbf{pista de auditoría} que permite rastrear cómo se derivaron las interpretaciones y conclusiones directamente de los datos recolectados, minimizando el sesgo del investigador y asegurando que los resultados reflejen fielmente las voces de los participantes.
