% !TEX root = main.tex


This study evaluates the implementation of PLANEAUML, an innovative web system designed for the digitization and optimization of academic planning at Martin Luther University, Jalapa Campus. The research addresses the issues of traditional manual methods, which caused inefficiencies and administrative overload. A Mixed Sequential Explanatory Design was employed, combining initial surveys with in-depth interviews conducted six months after implementation. The system, developed with Django and modern technologies, integrates artificial intelligence to assist in syllabus generation.

Initial quantitative results showed a drastic reduction in plan preparation times, with 81\% of teachers completing the task in less than an hour. The six-month qualitative follow-up revealed profound findings: system adoption was sustained and consolidated, transforming the organizational culture towards information centralization and standardization. It was identified that the saved time was not converted into leisure but reinvested in improving pedagogical quality, enabling more dynamic classes and personalized feedback. Furthermore, artificial intelligence was adopted not as a replacement, but as a "cognitive assistant" to overcome creative blocks and structure content.

It is concluded that PLANEAUML not only optimized administrative efficiency but also catalyzed an improvement in educational quality and teacher well-being. The study validates the efficacy of user-centered digitization and offers a replicable model for management modernization in higher education institutions.

Keywords: web system, digitization, academic planning, optimization, artificial intelligence, higher education management.