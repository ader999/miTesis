% !TEX root = main.tex


This study evaluates the implementation of PLANEAUML, a web system for the digitization and optimization of academic planning at Martin Luther University, Jalapa Campus, addressing the inefficiencies of manual methods. A Mixed Sequential Explanatory Design was employed, combining initial surveys with subsequent interviews. The system, developed with Django and AI, assists in syllabus generation.

Results showed a drastic reduction in planning times, with 81\% of teachers completing the task in less than an hour. Qualitatively, sustained adoption was observed, transforming organizational culture towards standardization. Saved time was reinvested in improving pedagogical quality and feedback. AI was adopted as a "cognitive assistant" to structure content, not as a replacement.

It is concluded that PLANEAUML optimized administrative efficiency and improved educational quality. The study validates user-centered digitization as a replicable model for higher education management.

Keywords: web system, digitization, academic planning, optimization, artificial intelligence, higher education management.