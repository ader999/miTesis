% !TEX root = main.tex

La presente encuesta se aplicó con el objetivo de diagnosticar la situación actual del proceso de planeación académica en la Universidad Martín Lutero Sede Jalapa. A través de este instrumento, se buscó identificar las herramientas empleadas, el tiempo invertido en la elaboración de sílabos y las principales dificultades que enfrentan los docentes en su labor de planificación.

\subsection{Encuesta 1: Evaluación del Proceso de Planeación Académica usado en la Universidad Martín Lutero Sede Jalapa}

\textbf{1. ¿Cuánto tiempo promedio dedica actualmente a la elaboración de un plan de estudio?}
\begin{itemize}
    \item[$\square$] Menos de 2 horas.
    \item[$\square$] Entre 2 y 5 horas.
    \item[$\square$] Más de 5 horas.
\end{itemize}

\textbf{2. ¿Qué herramientas utiliza principalmente para crear los planes de estudio y sílabos?}
\begin{itemize}
    \item[$\square$] Word o Excel.
    \item[$\square$] Sistema manual en papel.
    \item[$\square$] Otro (especificar): \underline{\hspace{5cm}}
\end{itemize}

\textbf{3. ¿Qué tan fácil considera cumplir con los plazos establecidos para la entrega de sílabos?}
\begin{itemize}
    \item[$\square$] Muy fácil.
    \item[$\square$] Fácil.
    \item[$\square$] Difícil.
    \item[$\square$] Muy difícil.
\end{itemize}

\textbf{4. ¿Con qué frecuencia enfrenta problemas como errores o retrabajo en la planeación académica?}
\begin{itemize}
    \item[$\square$] Nunca.
    \item[$\square$] Ocasionalmente.
    \item[$\square$] Frecuentemente.
    \item[$\square$] Siempre.
\end{itemize}

\textbf{5. ¿Cuáles son los principales problemas que enfrenta durante la planeación académica? (Puede marcar más de una opción)}
\begin{itemize}
    \item[$\square$] Falta de tiempo
    \item[$\square$] Errores frecuentes o retrabajo
    \item[$\square$] Dificultades técnicas con las herramientas actuales
    \item[$\square$] Coordinación con colegas o autoridades
    \item[$\square$] Otras (especifique): \underline{\hspace{5cm}}
\end{itemize}

\textbf{6. ¿Cuánto tiempo promedio invierte en corregir errores o ajustes en los planes de estudio ya elaborados?}
\begin{itemize}
    \item[$\square$] Menos de 1 hora
    \item[$\square$] Entre 1 y 3 horas
    \item[$\square$] Más de 3 horas
    \item[$\square$] No realizo correcciones
\end{itemize}

\textbf{7. ¿Qué aspectos del proceso de planeación académica le resultan más complicados? (Puede marcar más de una opción)}
\begin{itemize}
    \item[$\square$] Identificar los contenidos clave para cada asignatura
    \item[$\square$] Ajustarse a las directrices institucionales
    \item[$\square$] Cumplir con los plazos de entrega
    \item[$\square$] Otro (especifique): \underline{\hspace{5cm}}
\end{itemize}

\textbf{8. ¿Qué tan satisfecho está con las herramientas que utiliza actualmente para elaborar los sílabos y generar los planes de clase?}
\begin{itemize}
    \item[$\square$] Muy satisfecho
    \item[$\square$] Satisfecho
    \item[$\square$] Insatisfecho
    \item[$\square$] Muy insatisfecho
\end{itemize}

\textbf{9. ¿Cómo calificaría la coordinación con otros docentes o autoridades durante la planeación académica?}
\begin{itemize}
    \item[$\square$] Muy buena.
    \item[$\square$] Buena.
    \item[$\square$] Regular.
    \item[$\square$] Mala.
\end{itemize}

\textbf{10. ¿Qué tan útil considera recibir capacitaciones periódicas sobre nuevas herramientas o estrategias de planeación académica?}
\begin{itemize}
    \item[$\square$] Muy útil
    \item[$\square$] Útil
    \item[$\square$] Poco útil
    \item[$\square$] Nada útil
\end{itemize}

\textbf{11. ¿Con qué frecuencia utiliza recursos externos (como guías, manuales o ejemplos) para elaborar sílabos?}
\begin{itemize}
    \item[$\square$] Siempre
    \item[$\square$] Frecuentemente
    \item[$\square$] Ocasionalmente
    \item[$\square$] Nunca
\end{itemize}

\textbf{12. ¿Le gustaría contar con un sistema automatizado que simplifique la planeación académica?}
\begin{itemize}
    \item[$\square$] Sí, definitivamente.
    \item[$\square$] Podría ser útil.
    \item[$\square$] No estoy interesado.
\end{itemize}

Esta encuesta fue diseñada para evaluar la funcionalidad y aceptación del sistema PLANEAUML tras su implementación. El instrumento recopila la percepción de docentes y administrativos sobre la facilidad de uso, la eficiencia del sistema y el impacto de la automatización en la reducción de tiempos y errores durante el proceso de planificación académica.

\subsection{Encuesta 2: Evaluación del sistema PLANEAUML para la planeación docente}

\subsubsection{Sección 1: Información General y Rol}

\textbf{1. ¿Está familiarizado con el sistema PLANEAUML?}
\begin{itemize}
    \item[$\square$] Sí
    \item[$\square$] No
\end{itemize}
\textit{(Si responde ``No'', puede finalizar la encuesta aquí. ¡Gracias por su tiempo!)}

\textbf{2. ¿Cuál es su puesto/rol principal en la universidad?}
\begin{itemize}
    \item[$\square$] Administrativo
    \item[$\square$] Maestro (Docente)
\end{itemize}

\subsubsection{Sección 2: Adopción y Funcionalidad del Sistema}

\textbf{Sección 2A: Para Personal Administrativo}

\textit{(Responda esta sección ÚNICAMENTE si su rol es Administrativo)}

\textbf{3. ¿Cómo evaluaría la facilidad de uso general del sistema PLANEAUML desde la perspectiva administrativa?}
\begin{itemize}
    \item[$\square$] Muy fácil
    \item[$\square$] Fácil
    \item[$\square$] Complicado
    \item[$\square$] Muy complicado
\end{itemize}

\textbf{4. Desde su perspectiva, ¿cree que el panel de control de PLANEAUML permite gestionar de manera eficiente a los usuarios y sus roles?}
\begin{itemize}
    \item[$\square$] Sí
    \item[$\square$] Tal vez (Con algunas limitaciones)
    \item[$\square$] No
\end{itemize}

\textbf{5. ¿Qué aspectos del sistema, enfocados en la administración, le parecen más útiles? (Puede marcar más de una opción)}
\begin{itemize}
    \item[$\square$] Administración y control de permisos de usuarios
    \item[$\square$] Funcionalidades de exportación de datos
    \item[$\square$] Seguimiento del estado de los planes de estudio (ej. terminados, en proceso)
    \item[$\square$] Filtros para la búsqueda y organización de datos
    \item[$\square$] Otro (especificar): \underline{\hspace{5cm}}
\end{itemize}

\textbf{6. ¿Le parece eficiente la manera en que PLANEAUML maneja la estructura y organización de los datos académicos?}
\begin{itemize}
    \item[$\square$] Sí
    \item[$\square$] Tal vez
    \item[$\square$] No
\end{itemize}

\textbf{7. ¿Le parecen eficientes los filtros aplicados en el panel de control para la gestión y consulta de información?}
\begin{itemize}
    \item[$\square$] Sí
    \item[$\square$] Tal vez
    \item[$\square$] No
\end{itemize}

\textbf{8. En general, ¿cree que PLANEAUML es un sistema adecuado para gestionar los planes de estudio académicos de la institución?}
\begin{itemize}
    \item[$\square$] Sí
    \item[$\square$] Tal vez
    \item[$\square$] No
\end{itemize}

\textbf{Sección 2B: Para Maestros (Docentes)}

\textit{(Responda esta sección ÚNICAMENTE si su rol es Maestro/Docente)}

\textbf{9. ¿Cómo calificaría su experiencia general utilizando el sistema PLANEAUML para la planeación académica?}
\begin{itemize}
    \item[$\square$] Muy buena (Satisface completamente las necesidades)
    \item[$\square$] Buena (Funcional, pero con áreas de mejora)
    \item[$\square$] Regular (Satisface parcialmente las necesidades)
    \item[$\square$] No satisface las necesidades
\end{itemize}

\textbf{10. ¿Qué aspectos del sistema PLANEAUML le parecen más útiles para su labor docente? (Puede marcar más de una opción)}
\begin{itemize}
    \item[$\square$] Facilidad para completar el formulario del sílabo
    \item[$\square$] Mejor organización y control de sus datos de planeación
    \item[$\square$] Generación automática de contenido para el sílabo mediante Inteligencia Artificial (IA)
    \item[$\square$] Llenado automático del Plan de Clase y la Guía de Estudio (si aplica)
    \item[$\square$] Otro (especificar): \underline{\hspace{5cm}}
\end{itemize}

\textbf{11. ¿Cómo evaluaría la facilidad de uso del sistema PLANEAUML para elaborar sus planes de estudio y sílabos?}
\begin{itemize}
    \item[$\square$] Muy fácil
    \item[$\square$] Fácil
    \item[$\square$] Algo complejo
    \item[$\square$] Muy difícil
\end{itemize}

\textbf{12. Utilizando PLANEAUML, ¿cuánto tiempo promedio dedica ahora a elaborar un plan de estudios o sílabo completo?}
\begin{itemize}
    \item[$\square$] Menos de 1 hora
    \item[$\square$] Entre 1 y 2 horas (ej. 2 horas)
    \item[$\square$] Más de 2 horas (ej. más de 3 horas)
\end{itemize}

\textbf{13. ¿Considera que el sistema PLANEAUML reduce los tiempos de planeación en comparación con la forma tradicional (manual o con otras herramientas)?}
\begin{itemize}
    \item[$\square$] Sí
    \item[$\square$] No
\end{itemize}

\textbf{14. ¿Considera que el sistema PLANEAUML ayuda a reducir errores en la elaboración de sílabos y planes de estudio?}
\begin{itemize}
    \item[$\square$] Sí
    \item[$\square$] No
\end{itemize}

\textbf{15. ¿Cómo calificaría la funcionalidad de generación automática de contenido para sílabos mediante el uso de inteligencia artificial en PLANEAUML?}
\begin{itemize}
    \item[$\square$] Muy buena
    \item[$\square$] Buena
\item[$\square$] Regular
    \item[$\square$] Mala
    \item[$\square$] No he utilizado esta funcionalidad / No aplica
\end{itemize}

\textbf{16. ¿Cree que PLANEAUML puede contribuir significativamente a mejorar la calidad y eficiencia de la planificación académica de los docentes?}
\begin{itemize}
    \item[$\square$] Sí
    \item[$\square$] Tal vez
    \item[$\square$] No
\end{itemize}

\subsection{Entrevista a Docentes sobre la Experiencia con el Sistema PLANEAUML a Seis Meses de su Implementación}

El objetivo de esta entrevista es profundizar en la experiencia de los docentes que han utilizado el sistema PLANEAUML durante los últimos seis meses, para comprender a fondo su impacto a mediano plazo en la planificación académica y la práctica docente.

\subsubsection{Sección 1: Experiencia General y Adopción}

\textbf{1. Para empezar, ¿podría describir en sus propias palabras cómo ha sido su experiencia general utilizando PLANEAUML durante estos seis meses?}

\vspace{0.5cm}
\noindent\textit{(Espacio para respuesta abierta)}
\vspace{0.5cm}

\textbf{2. Recordando la forma en que planificaba antes (usando Word, Excel, etc.), ¿cuál es el cambio más significativo que ha notado en su rutina de trabajo desde que adoptó PLANEAUML?}

\vspace{0.5cm}
\noindent\textit{(Espacio para respuesta abierta)}
\vspace{0.5cm}

\textbf{3. En una escala del 1 al 10, donde 1 es ``nada integrado'' y 10 es ``completamente integrado'', ¿qué tan parte de su rutina de planificación diría que es PLANEAUML hoy en día? ¿Qué factores influyen en ese nivel de integración?}

\vspace{0.5cm}
\noindent\textit{(Espacio para respuesta abierta)}
\vspace{0.5cm}

\subsubsection{Sección 2: Impacto en la Eficiencia y el Flujo de Trabajo}

\textbf{4. La investigación inicial mostró una reducción drástica en el tiempo de elaboración de los planes de estudio. Seis meses después, ¿siente que esa eficiencia se ha mantenido? ¿Ha descubierto nuevas formas de ahorrar tiempo con el sistema a medida que se familiariza más con él?}

\vspace{0.5cm}
\noindent\textit{(Espacio para respuesta abierta)}
\vspace{0.5cm}

\textbf{5. El estudio destacó la ``Facilidad del llenado del formulario del sílabo'' como un aspecto muy valorado. ¿Podría detallar qué es lo que hace que este proceso sea más sencillo en comparación con el método anterior?}

\vspace{0.5cm}
\noindent\textit{(Espacio para respuesta abierta)}
\vspace{0.5cm}

\textbf{6. Más allá de la creación inicial, ¿cómo ha impactado PLANEAUML en tareas como la corrección de errores, la realización de ajustes a los sílabos o la reutilización de material de semestres anteriores?}

\vspace{0.5cm}
\noindent\textit{(Espacio para respuesta abierta)}
\vspace{0.5cm}

\subsubsection{Sección 3: Usabilidad y Experiencia de Usuario a Mediano Plazo}

\textbf{7. Al principio, todo el personal calificó el sistema como ``Fácil'' o ``Muy Fácil''. Ahora que ha pasado más tiempo, ¿ha encontrado alguna complejidad o dificultad que no fue evidente al inicio?}

\vspace{0.5cm}
\noindent\textit{(Espacio para respuesta abierta)}
\vspace{0.5cm}

\textbf{8. ¿Hay alguna parte del sistema que le resulte particularmente intuitiva o, por el contrario, alguna que considere que podría ser más clara o sencilla de navegar?}

\vspace{0.5cm}
\noindent\textit{(Espacio para respuesta abierta)}
\vspace{0.5cm}

\textbf{9. Desde una perspectiva de organización, las encuestas anteriores mencionan que los docentes valoraron tener un ``Mejor control de sus datos''. ¿Podría darnos un ejemplo práctico de cómo el sistema le ha ayudado a sentirse más organizado con su información académica?}

\vspace{0.5cm}
\noindent\textit{(Espacio para respuesta abierta)}
\vspace{0.5cm}

\subsubsection{Sección 4: Retroalimentación sobre Funcionalidades Específicas}

\textbf{10. Una de las características más innovadoras es la generación de contenido de sílabos con Inteligencia Artificial. ¿Con qué frecuencia utiliza esta función en su día a día?}

\vspace{0.5cm}
\noindent\textit{(Espacio para respuesta abierta)}
\vspace{0.5cm}

\textbf{11. ¿Cómo describiría el rol que juega la IA en su proceso? ¿La usa como un punto de partida, para generar ideas, o para redactar secciones completas? ¿Qué tan útil es el contenido que genera para su asignatura específica?}

\vspace{0.5cm}
\noindent\textit{(Espacio para respuesta abierta)}
\vspace{0.5cm}

\textbf{12. Pensando en la exportación de documentos (Word y Excel), ¿qué tan útiles le han resultado estas funciones? ¿Los formatos generados cumplen con lo que necesita para sus reportes o para compartir información?}

\vspace{0.5cm}
\noindent\textit{(Espacio para respuesta abierta)}
\vspace{0.5cm}

\subsubsection{Sección 5: Impacto en la Calidad de la Planificación y la Docencia}

\textbf{13. El objetivo final de optimizar la planificación es mejorar la calidad educativa. ¿Siente que el tiempo y el esfuerzo que ha ahorrado con PLANEAUML le ha permitido enfocarse más en otros aspectos de su docencia, como preparar mejores clases, buscar nuevos recursos o dar una retroalimentación más detallada a los estudiantes?}

\vspace{0.5cm}
\noindent\textit{(Espacio para respuesta abierta)}
\vspace{0.5cm}

\textbf{14. ¿Cree que el uso de una plataforma estandarizada ha mejorado la coherencia y la calidad general de los planes de estudio en toda la universidad? ¿Por qué?}

\vspace{0.5cm}
\noindent\textit{(Espacio para respuesta abierta)}
\vspace{0.5cm}

\subsubsection{Sección 6: Sugerencias y Visión a Futuro}

\textbf{15. Si tuviera la oportunidad de solicitar una nueva funcionalidad o una mejora para PLANEAUML, ¿cuál sería su principal petición?}

\vspace{0.5cm}
\noindent\textit{(Espacio para respuesta abierta)}
\vspace{0.5cm}

\textbf{16. Pensando en el futuro, ¿qué otro proceso académico de la universidad cree que podría beneficiarse de una digitalización similar a la que PLANEAUML ha traído a la planificación de estudios?}

\vspace{0.5cm}
\noindent\textit{(Espacio para respuesta abierta)}
\vspace{0.5cm}

\textbf{17. Para concluir, ¿cuál sería su recomendación para un docente de otra universidad que estuviera considerando adoptar un sistema como PLANEAUML?}

\vspace{0.5cm}
\noindent\textit{(Espacio para respuesta abierta)}
\vspace{0.5cm}

\clearpage
\subsection{Cronograma de Actividades}

\begin{longtable}{p{5cm} p{2.5cm} p{3cm} p{1.5cm} p{2.5cm}}
\caption{Cronograma de Actividades} \label{tab:cronograma} \\
\toprule
\textbf{Actividad} & \textbf{Fecha} & \textbf{Lugar} & \textbf{Hora} & \textbf{Responsable} \\
\midrule
\endfirsthead
\toprule
\textbf{Actividad} & \textbf{Fecha} & \textbf{Lugar} & \textbf{Hora} & \textbf{Responsable} \\
\midrule
\endhead
Desarrollar tema del proyecto de tesis & 30/08/2024 & Comunidad Trapiche & 10 am & Ader Zeas \\
Desarrollo del tema, objetivos y justificación. & 30/08/2024 & Comunidad Trapiche & 10 am & Ader Zeas \\
Desarrollo del marco teórico & 01/11/2024 & Comunidad Trapiche & 10am & Ader Zeas \\
Elaboración de Hipótesis y Introducción & 08/11/2024 & Comunidad Trapiche & 10am & Ader Zeas \\
Investigación de los Antecedentes & 16/11/2024 & Comunidad Trapiche & 9am & Ader Zeas \\
Elaborar Encuestas para la recolección de datos & 25/11/2024 & Comunidad Trapiche & 9pm & Ader Zeas \\
Desarrollo de la metodología & 28/11/2024 & Comunidad Trapiche & 10am & Ader Zeas \\
Conclusión y recomendaciones & 15/01/2025 & Comunidad Trapiche & 8pm & Ader Zeas \\
Revisiones y Ajustes & 27/06/2025 & Comunidad Trapiche & 11am & Ader Zeas \\
Revisiones y ajustes de la predefensa & 24/11/2025 & Comunidad & 8pm & Ader Zeas \\
\bottomrule
\end{longtable}