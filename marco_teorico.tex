% !TEX root = main.tex

\subsection{Planeación didáctica}

La planeación didáctica o programación docente es el proceso a través del cual el docente toma una serie de decisiones y realiza un conjunto de operaciones para aplicar el programa establecido institucionalmente de forma concreta y específica en actividades didácticas.

De esta forma, el programa delineado institucionalmente no se aplica de manera cerrada, sino que sirve de referencia al tiempo que se adapta, al contexto y realidad particular, teniendo en cuenta los objetivos, las características de los alumnos y los contenidos, entre otros factores.

En la planeación curricular se describen de forma clara y específica las actividades a realizar y las estrategias para lograr los objetivos de una manera intencionada y organizada, por lo que se convierte en una forma de orientar los procesos que se llevarán a cabo en el aula.

Los sistemas educativos de cada país se establecen de forma diferente, tanto en estructura como en función: en cada país variarán aspectos como la flexibilidad permitida, el alcance, los elementos mínimos necesarios, entre otros factores. Por esto es importante considerar las bases legales asociadas a la planeación didáctica en el país correspondiente. \cite{cajal2020}

\subsubsection{Características de la planeación didáctica}

Las planeaciones didácticas deben tener una serie de características para que puedan cumplir con sus objetivos:

Deben estar por escrito y deben presentarse de forma estructurada los objetivos y las técnicas para llevarlos a cabo.

Deben partir siempre del programa o marco formativo institucional.

Se debe hacer de forma coordinada con los demás docentes, de modo que reduzca la incertidumbre al saber todos hacia qué se trabaja y cómo se llegará.

Es un instrumento que debe ser flexible, ya que no todo se puede prever, y debe estar abierta a cualquier mejora que se pueda realizar.

Debe adecuarse al contexto específico, por lo que debe personalizarse según la realidad actual.

Debe ser realista, de modo que su aplicación pueda ser viable.

\subsection{Las TIC en la planeación}

La incorporación de las Tecnologías de la Información y Comunicación (TIC) en la educación es un proceso inminente y acelerado a nivel mundial. Sin embargo, su éxito no depende de la mera dotación de equipos, sino de una planificación intencionada y estructurada en todos los niveles del sistema educativo. Esta planificación abarca desde la estrategia global del centro educativo hasta la planeación didáctica que cada docente diseña para su aula.

\subsubsection{La Necesidad de una Planificación Estratégica Institucional}

Antes de que las TIC lleguen al aula, su integración debe ser concebida como parte del proyecto educativo del centro. Las instituciones recurren a la planificación estratégica para gestionar recursos, definir objetivos y metodologías, y anticiparse a los desafíos y oportunidades que presenta la tecnología \cite{laia2014}. La improvisación puede resolver situaciones inmediatas, pero no garantiza resultados medibles ni sostenibles a largo plazo.

\paragraph{Principios Clave de la Planificación Estratégica}

Los expertos coinciden en que la planificación es un elemento clave para el éxito de cualquier innovación educativa apoyada en las TIC. Sin una planificación adecuada de herramientas, fines y actividades, se pueden generar "situaciones de enseñanza un tanto erráticas, e incluso caóticas, provocando cierta frustración en el profesorado" \textcite{laia2014}. Por ello, la estrategia debe seguir principios fundamentales:

La Pedagogía sobre la Tecnología: Lo más importante es que las TIC se utilicen al servicio de un modelo educativo claro. Primero se debe pensar en la pedagogía y luego en la tecnología, y no al revés. Sin un modelo pedagógico, las TIC pierden su valor transformador \cite{laia2014}.

El Docente como Pieza Clave: El profesorado es fundamental en este proceso. Por ello, la incorporación de las TIC debe ser un proceso motivador con objetivos definidos, la formación docente es indispensable y las tecnologías deben adaptarse al proyecto educativo, no a la inversa \cite{laia2014}.

Proceso Participativo y Progresivo: La planificación ideal es un proceso dinámico que involucra a toda la comunidad educativa (directivos, docentes, familias). Además, se recomienda realizar un cambio progresivo en lugar de una implementación radical, para poder detectar errores y ajustar el proceso sobre la marcha \cite{laia2014}.

\subsubsection{El Impacto de las TIC en la Planeación Didáctica del Docente}

Una vez definida la estrategia institucional, el verdadero desafío se traslada al aula. Para \textcite{diazangel2013}, el reto fundamental es "construir un uso educativo y, en estricto sentido, didáctico" de las TIC. Esto implica transformar la planeación de la enseñanza para ir más allá de la simple transmisión de información.

\paragraph{De la Clase Frontal a los Ambientes de Aprendizaje}

La integración de las TIC exige abandonar el modelo tradicional de la clase frontal. En su lugar, el docente debe diseñar ambientes de aprendizaje donde existen múltiples flujos de información (videos, documentos, simulaciones) y cada alumno puede generar sus propias fuentes de consulta \textcite{diazangel2013}. Esto fomenta un pensamiento divergente, donde cada estudiante explora distintas rutas de información. La labor del docente en su planeación es, entonces, orientar y hacer converger estos esfuerzos hacia un objetivo común mediante el diálogo y el trabajo colaborativo.

\paragraph{La Secuencia Didáctica como Herramienta Central de Planeación}

La herramienta clave del docente para crear estos ambientes es la secuencia didáctica, definida como un conjunto de actividades de aprendizaje planificadas que orientan la tarea de aprender \textcite{diazangel2013}. Una secuencia didáctica que incorpora TIC se estructura en tres momentos:

a) Actividades de Apertura: Tienen como propósito crear una expectación, un enigma o una interrogante que motive al estudiante. Se parte de un caso o problema que active los conocimientos previos y lo invite a formularse preguntas. Las TIC, como un video o un foro inicial, pueden ser un excelente recurso para presentar el problema \cite{diazangel2013}.

b) Actividades de Desarrollo: Su objetivo es acercar al estudiante a la información para que interactúe con ella y construya conocimiento. El docente debe planear tareas complejas que exijan más que "copiar y pegar". Se trata de "movilizar" la información: analizarla, sintetizarla y articularla con otros saberes para construir una estructura conceptual propia. Esto implica guiar a los alumnos en el uso de bases de datos, buscadores académicos y otras fuentes rigurosas \cite{diazangel2013}.

c) Actividades de Cierre: Permiten al estudiante reorganizar y sintetizar la información que ha integrado. La planeación debe incluir actividades de síntesis, como la elaboración de un informe, un debate, la exposición de un proyecto o la resolución final del problema planteado. Las TIC facilitan la creación de portafolios digitales, blogs o presentaciones multimedia como evidencia de este proceso \cite{diazangel2013}.

\paragraph{La Evolución de la Evaluación: Hacia una Evaluación para el Aprendizaje}

La planeación con TIC también debe transformar la evaluación. Se debe transitar de una evaluación del aprendizaje (centrada en la calificación final) a una evaluación para el aprendizaje (enfocada en la retroalimentación continua) \textcite{diazangel2013}. Las dificultades y avances observados en cada etapa de la secuencia didáctica sirven como insumo para que el docente ajuste su planeación y para que el alumno comprenda su propio proceso. Las evidencias generadas a través de las TIC se convierten en una parte fundamental de un portafolio que refleja el proceso completo de construcción de conocimiento.

