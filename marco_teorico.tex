% !TEX root = main.tex

\subsection{Planeación didáctica}

La planeación didáctica o programación docente es el proceso a través del cual el docente toma una serie de decisiones y realiza un conjunto de operaciones para aplicar el programa establecido institucionalmente de forma concreta y específica en actividades didácticas.

De esta forma, el programa delineado institucionalmente no se aplica de manera cerrada, sino que sirve de referencia al tiempo que se adapta, al contexto y realidad particular, teniendo en cuenta los objetivos, las características de los alumnos y los contenidos, entre otros factores.

En la planeación curricular se describen de forma clara y específica las actividades a realizar y las estrategias para lograr los objetivos de una manera intencionada y organizada, por lo que se convierte en una forma de orientar los procesos que se llevarán a cabo en el aula.

Los sistemas educativos de cada país se establecen de forma diferente, tanto en estructura como en función: en cada país variarán aspectos como la flexibilidad permitida, el alcance, los elementos mínimos necesarios, entre otros factores. Por esto es importante considerar las bases legales asociadas a la planeación didáctica en el país correspondiente. \cite{cajal2020}

\subsubsection{Características de la planeación didáctica}

Las planeaciones didácticas deben tener una serie de características para que puedan cumplir con sus objetivos:

Deben estar por escrito y deben presentarse de forma estructurada los objetivos y las técnicas para llevarlos a cabo.

Deben partir siempre del programa o marco formativo institucional.

Se debe hacer de forma coordinada con los demás docentes, de modo que reduzca la incertidumbre al saber todos hacia qué se trabaja y cómo se llegará.

Es un instrumento que debe ser flexible, ya que no todo se puede prever, y debe estar abierta a cualquier mejora que se pueda realizar.

Debe adecuarse al contexto específico, por lo que debe personalizarse según la realidad actual.

Debe ser realista, de modo que su aplicación pueda ser viable.

\subsection{Las TIC en la planeación}

La incorporación de las Tecnologías de la Información y Comunicación (TIC) en la educación es un proceso inminente y acelerado a nivel mundial. Sin embargo, su éxito no depende de la mera dotación de equipos, sino de una planificación intencionada y estructurada en todos los niveles del sistema educativo. Esta planificación abarca desde la estrategia global del centro educativo hasta la planeación didáctica que cada docente diseña para su aula.

\subsubsection{La Necesidad de una Planificación Estratégica Institucional}

Antes de que las TIC lleguen al aula, su integración debe ser concebida como parte del proyecto educativo del centro. Las instituciones recurren a la planificación estratégica para gestionar recursos, definir objetivos y metodologías, y anticiparse a los desafíos y oportunidades que presenta la tecnología \cite{laia2014}. La improvisación puede resolver situaciones inmediatas, pero no garantiza resultados medibles ni sostenibles a largo plazo.

\paragraph{Principios Clave de la Planificación Estratégica}

Los expertos coinciden en que la planificación es un elemento clave para el éxito de cualquier innovación educativa apoyada en las TIC. Sin una planificación adecuada de herramientas, fines y actividades, se pueden generar "situaciones de enseñanza un tanto erráticas, e incluso caóticas, provocando cierta frustración en el profesorado" \textcite{laia2014}. Por ello, la estrategia debe seguir principios fundamentales:

La Pedagogía sobre la Tecnología: Lo más importante es que las TIC se utilicen al servicio de un modelo educativo claro. Primero se debe pensar en la pedagogía y luego en la tecnología, y no al revés. Sin un modelo pedagógico, las TIC pierden su valor transformador \cite{laia2014}.

El Docente como Pieza Clave: El profesorado es fundamental en este proceso. Por ello, la incorporación de las TIC debe ser un proceso motivador con objetivos definidos, la formación docente es indispensable y las tecnologías deben adaptarse al proyecto educativo, no a la inversa \cite{laia2014}.

Proceso Participativo y Progresivo: La planificación ideal es un proceso dinámico que involucra a toda la comunidad educativa (directivos, docentes, familias). Además, se recomienda realizar un cambio progresivo en lugar de una implementación radical, para poder detectar errores y ajustar el proceso sobre la marcha \cite{laia2014}.

\subsubsection{El Impacto de las TIC en la Planeación Didáctica del Docente}

Una vez definida la estrategia institucional, el verdadero desafío se traslada al aula. Para \textcite{diazangel2013}, el reto fundamental es "construir un uso educativo y, en estricto sentido, didáctico" de las TIC. Esto implica transformar la planeación de la enseñanza para ir más allá de la simple transmisión de información.

\paragraph{De la Clase Frontal a los Ambientes de Aprendizaje}

La integración de las TIC exige abandonar el modelo tradicional de la clase frontal. En su lugar, el docente debe diseñar ambientes de aprendizaje donde existen múltiples flujos de información (videos, documentos, simulaciones) y cada alumno puede generar sus propias fuentes de consulta \textcite{diazangel2013}. Esto fomenta un pensamiento divergente, donde cada estudiante explora distintas rutas de información. La labor del docente en su planeación es, entonces, orientar y hacer converger estos esfuerzos hacia un objetivo común mediante el diálogo y el trabajo colaborativo.

\paragraph{La Secuencia Didáctica como Herramienta Central de Planeación}

La herramienta clave del docente para crear estos ambientes es la secuencia didáctica, definida como un conjunto de actividades de aprendizaje planificadas que orientan la tarea de aprender \textcite{diazangel2013}. Una secuencia didáctica que incorpora TIC se estructura en tres momentos:

a) Actividades de Apertura: Tienen como propósito crear una expectación, un enigma o una interrogante que motive al estudiante. Se parte de un caso o problema que active los conocimientos previos y lo invite a formularse preguntas. Las TIC, como un video o un foro inicial, pueden ser un excelente recurso para presentar el problema \cite{diazangel2013}.

b) Actividades de Desarrollo: Su objetivo es acercar al estudiante a la información para que interactúe con ella y construya conocimiento. El docente debe planear tareas complejas que exijan más que "copiar y pegar". Se trata de "movilizar" la información: analizarla, sintetizarla y articularla con otros saberes para construir una estructura conceptual propia. Esto implica guiar a los alumnos en el uso de bases de datos, buscadores académicos y otras fuentes rigurosas \cite{diazangel2013}.

c) Actividades de Cierre: Permiten al estudiante reorganizar y sintetizar la información que ha integrado. La planeación debe incluir actividades de síntesis, como la elaboración de un informe, un debate, la exposición de un proyecto o la resolución final del problema planteado. Las TIC facilitan la creación de portafolios digitales, blogs o presentaciones multimedia como evidencia de este proceso \cite{diazangel2013}.

\paragraph{La Evolución de la Evaluación: Hacia una Evaluación para el Aprendizaje}

La planeación con TIC también debe transformar la evaluación. Se debe transitar de una evaluación del aprendizaje (centrada en la calificación final) a una evaluación para el aprendizaje (enfocada en la retroalimentación continua) \textcite{diazangel2013}. Las dificultades y avances observados en cada etapa de la secuencia didáctica sirven como insumo para que el docente ajuste su planeación y para que el alumno comprenda su propio proceso. Las evidencias generadas a través de las TIC se convierten en una parte fundamental de un portafolio que refleja el proceso completo de construcción de conocimiento.

\subsection{Modelo TPACK}

El TPACK es un modelo de enseñanza y aprendizaje que identifica los tipos de conocimiento que un docente necesita dominar para integrar las Tecnologías de la Información y la Comunicación (TIC) de una forma eficaz en la enseñanza que imparte. Se incluye entre los modelos cognitivos en ambientes cooperativos donde, además, se utiliza la tecnología. El modelo TPACK tiene en cuenta el hecho de que la tecnología ha llegado para quedarse, mientras repetido por diferentes autores que tratan la incursión de las TIC en el aula. Ante esta realidad se propone al profesorado, no solo el aprendizaje en el uso de las tecnologías sino también formarse en habilidades para adaptarse a los cambios que se produzcan ante nuevos software y hardware. \textcite{oscarDanilo2021}

\subsubsection{Desarrollo temático}

El TPACK es un modelo de enseñanza y aprendizaje que identifica los tipos de conocimiento que un docente necesita dominar para integrar las Tecnologías de la Información y la Comunicación (TIC) de una forma eficaz en la enseñanza que imparte. Se incluye entre los modelos cognitivos en ambientes cooperativos donde, además, se utiliza la tecnología.

De acuerdo a las siglas TPACK se hace mención al acrónimo de la expresión “Technological Pedagogical Content Knowledge, en la cual se destacan como pioneros de implementar este modelo de enseñanza aprendizaje a los profesores Punya, Mishra y Mattew J. Koehler, de la Universidad Estatal de Michigan (entre 2006 y 2009).

Diagrama modelo TPACK
En la imagen anterior se observan todos los tipos de conocimiento que provee el TPACK. Estos conocimientos no trabajan aislados, sino que están entrelazados entre ellos, generando nuevos conocimientos que emergen del modelo. En ese sentido dicho modelo engloba tres elementos que intervienen en la adquisición de conocimiento:

Content Knowledge (CK) conocimiento sobre el contenido de la materia concreta que se quiere enseñar.
Pedagogical Knowledge (PK) conocimiento de la pedagogía necesaria para que el alumnado alcance esos contenidos.
Technology Knowledg (TK) conocimiento de la tecnología que interviene en el proceso de aprendizaje.
Partiendo de estos tres componentes básicos se llevan a cabo diversas combinaciones de manera que se construye un entramado de interrelaciones que todo docente debe conocer y utilizar para una correcta integración de las TIC en su actividad diaria (Mishra y Koehler, 2006).

Este autor considera que el conocimiento del ámbito científico o materia de especialidad del profesor y su conocimiento pedagógico estaban, o podían estar, separados y debían ser unidos. De este modo, el conocimiento del contenido se refiere al que enseñar y el conocimiento pedagógico al cómo hacerlo.

Así la expresión: “conocimiento pedagógico del contenido” es diferente del conocimiento pedagógico sobre cómo enseñar en general, al tiempo que es distinto del saber de un área de terminada, de ser un experto en un determinado contenido, lo que no asegura que se sepa cómo enseñarlo. La expresión trata de combinar, o mejor interceptar, ambas dimensiones, convirtiéndose así en un conocimiento práctico sobre cómo enseñar lo que se supone que debe ser enseñado en una asignatura.  \textcite{oscarDanilo2021}

\subsubsection{Breve reseña del TPACK}

El modelo TPACK es una adecuación del modelo plateado por Shulman 1986, la cual lo llamo PCK, enmarcado dentro de las llamadas investigaciones sobre el pensamiento de los docentes, que fijan su atención en torno a la planificación que el profesorado hace de los contenidos a impartir y de las actividades propuestas para su consecución.

La propuesta PCK hace hincapié en dos componentes del proceso de enseñanza-aprendizaje que son los contenidos (CK) y la pedagogía (PK). A partir de ellos, presta una especial atención a cómo los contenidos de las materias concretas se organizan y adaptan a partir de la pedagogía para que lleguen adecuadamente al alumnado, es decir, como interactúan Contenido y Pedagogía.

Shulman comienza a despertar el interés por la revalorización de los estudios que prestan atención a cómo se unen la formación disciplinar y la didáctica específica, pero desde una nueva perspectiva, a través de los resultados del proyecto “Knowledge Growth in a Profession: development of knowledge in teaching” y la definición del “conocimiento base” con el que debe contar un profesional de la enseñanza (Bolívar, 1993).

Esta nueva manera de entender la comunión entre ambos elementos pasa por un proceso de “reflexión” del docente ante su labor de conjunción de contenidos disciplinares y contenidos pedagógicos. Y es en este contexto donde arranca el modelo propuesto en 2006, por P. Mishra y M.J. Koehler, bajo la denominación de “Technological Pedagogical Content Knowledge” o TPACK (En castellano Conocimiento Tecnológico, Pedagógico y del Contenido o Disciplinar).

Para el 2006 Julio Cabero profesor de la Universidad Estatal de Michigan de España, incorpora una nueva variable a la idea de Shulman (PCK) que es el “contenido tecnológico” (TK). Mishra y Koehler (2006) integran este TK al “contenido del conocimiento” (CK) y al “contenido pedagógico” (PK) y hacen hincapié en los diferentes tipos de conocimientos que los docentes necesitan para poder realizar la incorporación de las TIC de una manera correcta y eficaz con la finalidad de alcanzar “efectos significativos en el aprendizaje de sus alumnos” (Cabero et al., 2013).

De esta manera, el modelo TPACK lo conforman siete componentes (Baran et al, 2011). Tres que sería los conocimientos “base”:

1. Content Knowledge (CK) o Conocimiento sobre el contenido de la materia
2. Pedagogical Knowledge (PK) o Conocimiento Pedagógico
3. Technology Knowledge (TK) o Conocimiento Tecnológico
Los demás surgen a partir de la combinación de los tres conocimientos “base” anteriormente citados:

4. Pedagogical Content Knowledge (PCK) o Conocimiento Pedagógico del Contenido.
5. Technological Content Knowledge (TCK) o Conocimiento de la utilización de las Tecnologías.
6. Technological Pedagogical Knowledge (TPK) o Conocimiento Pedagógico Tecnológico.
En este sentido la conjunción de los seis conocimientos anteriores forma el llamado Technological Pedagogical Content Knowledge (TPACK), es decir, el Contenido Tecnológico, Pedagógico y del Contenido. (componente 7)

En consecuencia, el modelo propone que para que el docente cuente con la capacitación para incorporar las TIC en el aula necesita no sólo poseer los conocimientos “base” (contenidos 1, 2, 3) de una manera aislada e independiente, sino que también debe poseerlos en interacción (4, 5, 6 y 7). Sólo así, la tecnología se incorporará al proceso formativo de manera adecuada y logrará los objetivos de Enseñanza aprendizaje previstos por el discente.  \textcite{oscarDanilo2021}

\subsubsection{Tipos de conocimientos del modelo TPACK}

Este modelo fue propuesto por Mishra \& Koehler (2006) en el cual intervienen tres formas primarias de conocimiento que se entrelazan. El contenido, la Pedagogía y la Tecnología. El enfoque de este modelo va más allá de estas tres esferas, enfatizando los tipos de conocimiento que residen en sus intersecciones: El Conocimiento de Contenido Pedagógico, El Conocimiento de Contenido Tecnológico, el Conocimiento Tecnológico Pedagógico y el Conocimiento de Contenido Tecnológico Pedagógico.

La efectiva integración de la tecnología en la pedagogía de una materia específica requiere desarrollar sensibilidad para la relación dinámica y transaccional entre estos componentes del conocimiento situados en contextos únicos. Los docentes individualmente, el nivel de los estudiantes, los factores específicos de la escuela, la demografía, la cultura y otros factores aseguran que cada situación es única y que una sola combinación de contenido, tecnología y pedagogía aplicarán para cada maestro, para cada curso o para cada visión de enseñanza.

A continuación, se define cada tipo de conocimiento dentro del modelo:

\paragraph{Conocimiento de los Contenidos (CK)}

Es el conocimiento de los profesores sobre la materia que hay que aprender o enseñar. El contenido que se aborda en Ciencias en Secundaria o la Historia es diferente del contenido que se aborda en un curso de Universidad, o en el Arte o un seminario de postgrado en Astrofísica.

Como señaló Shulman (1986), este conocimiento podría incluir el conocimiento de los conceptos, teorías, ideas, marcos de organización, el conocimiento de evidencias y pruebas, así como las prácticas establecidas y enfoques hacia el desarrollo de tal conocimiento. Es el QUÉ se enseña.

\paragraph{Conocimiento Pedagógico (PK)}

Es el conocimiento profundo de los profesores sobre los procesos y las prácticas o métodos de enseñanza y aprendizaje. Abarca, entre otras cosas, los fines educativos en general, valores y objetivos.

Esta forma genérica de conocimiento se aplica a la comprensión de cómo aprenden los estudiantes, habilidades de manejo de la clase en general, la planificación de clases y la evaluación de los alumnos. Es el CÓMO se enseña.

\paragraph{El Conocimiento Tecnológico (TK)}

Es el conocimiento sobre ciertos modos de pensar y trabajar con la tecnología, las herramientas y los recursos. Trabajar con la tecnología se puede aplicar a todas las herramientas y recursos tecnológicos.

Esto incluye entender la tecnología de la información de forma lo suficientemente amplia como para aplicarla de manera productiva en el trabajo y en la vida cotidiana, ser capaz de reconocer cuándo la tecnología de la información puede ayudar u obstaculizar el logro de un objetivo, y ser capaz de adaptarse continuamente a los cambios de la misma.

\paragraph{El conocimiento Pedagógico del Contenido (PCK)}

El conocimiento de la pedagogía se aplica a la enseñanza de contenidos específicos.  Es central la noción de la transformación de la materia para la enseñanza.

En concreto, Shulman (1986), esta transformación se produce cuando el maestro interpreta la materia, encuentra varias maneras de representarla, y se adapta y adapta los materiales de instrucción a las concepciones alternativas y conocimientos previos de los alumnos.

PCK cubre la actividad principal de la enseñanza, el aprendizaje, el currículo, la evaluación y la presentación de informes, así como las condiciones que promueven el aprendizaje y los vínculos entre los planes de estudio, la evaluación y la pedagogía.

\paragraph{El Conocimiento Tecnológico del Contenido (TCK)}

La comprensión de la manera en que la tecnología y el contenido se influencian y limitan entre sí. Los profesores tienen que dominar más que la materia que enseñan; deben tener un profundo conocimiento de la manera en que el objeto (o los tipos de representaciones que se pueden construir) se pueden cambiar mediante la aplicación de tecnologías particulares.

Los maestros necesitan entender qué tecnologías específicas son las más adecuadas para abordar el aprendizaje objeto en sus dominios y cómo el contenido dicta o quizás incluso cambia la tecnología, o viceversa.

\paragraph{El Conocimiento Tecnológico-Pedagógico (TPK)}

Se refiere a la comprensión sobre cómo la enseñanza y el aprendizaje pueden cambiar cuando se utilizan determinadas tecnologías de manera particular. Esto incluye saber las posibilidades y limitaciones de una gama de herramientas tecnológicas y pedagógicas que se relacionan con diseños apropiados para el desarrollo y las estrategias pedagógicas.

\paragraph{Conocimiento de Contenido tecnológico Pedagógico (TPACK)}


Subyacente a una enseñanza significativa y profundamente competente con la tecnología, el TPACK es diferente del conocimiento de los tres conceptos en forma individual. En su lugar, TPACK es la base de la enseñanza efectiva con la tecnología, lo que requiere una comprensión de la representación de los conceptos que utilizan tecnologías; técnicas pedagógicas que utilizan tecnologías de manera constructiva para enseñar a los contenidos; el conocimiento de lo que hace fáciles o difíciles los conceptos que hay que aprender y cómo la tecnología puede ayudar a corregir algunos de los problemas que afrontan los estudiantes; conocimientos previos de los conocimientos y teorías epistemológicas de los estudiantes; y el conocimiento de cómo las tecnologías pueden ser utilizadas para construir el conocimiento existente para desarrollar nuevas epistemologías o fortalecer las ya existentes. \textcite{oscarDanilo2021}

\subsubsection{El interés por el estudio del conocimiento profesional docente}

Históricamente, se han clasificado los estudios sobre la enseñanza utilizando las categorías: proceso-producto, mediacional y ecológico. Es Angulo Rasco (1999) quien marcó el interés de agrupar estas corrientes en programas de investigación; plantea un pasaje que implica desplazar la pregunta sobre la enseñanza efectiva hacia la del conocimiento docente. El docente como pensador se ha instalado como un avance sobre los modelos conductistas, que lo ven más bien como un autómata.

Los conocimientos que necesita el profesorado para realizar su trabajo han sido estudiados hace mucho tiempo atrás. Este interés se vuelve a revitalizar a partir de ubicar la problemática en la integración de las TIC en la enseñanza. Previamente, uno de los primeros modelos desarrollados para estructurar el conocimiento del profesor desde una visión global fue propuesto por Shulman (1987). Quien se orientó a mostrar cómo la enseñanza efectiva necesita dominios específicos de conocimiento que se relacionan entre ellos. Hasta ese entonces, estos dominios eran considerados de manera aislada. Estudió el proceso de raciocinio pedagógico, del que los profesores derivan una forma especializada de conocimiento. Emerge la categoría Conocimiento Didáctico del Contenido (CDC), los profesores no solo necesitan el conocimiento didáctico y una sólida comprensión de los contenidos del programa de estudios, sino también estrategias y habilidades que se pueden aplicar tanto al estudiante como a la propia materia.

La aportación de Shulman significó un avance importante tanto en el campo profesional de la formación del profesorado como en el campo de la investigación; no obstante, su modelo no incluía de forma específica la integración de tecnología en el saber del profesorado. Esto fue así porque, en los tiempos en que se desarrolló el CDC, la tecnología era transparente, estable y de conocimiento general (Koheler et al., 2015). El panorama cambia con la irrupción de las tecnologías digitales. Incorporarlas a la enseñanza generó la necesidad de contemplar un nuevo dominio: el conocimiento tecnológico.

\begin{figure}
	\centering
	
	\caption[Esquema TPACK]{Modelo TPACK (Technological Pedagogical Content Knowledge). Adaptado de Mishra \& Koehler (2006).}
	\includegraphics[width=0.7\linewidth]{figuras/marco teorico/tpack1.png}
	\label{fig:tpack1}
\end{figure}

"Esto dio lugar a una corriente que proponía una expansión del modelo inicial para incluir un nuevo componente base (Figura 1). La intersección de estos tres conocimientos (contenido, didáctica y tecnología) habilita un nuevo y más importante dominio: el Conocimiento Didáctico Tecnológico del Contenido o CDTC. Para que la tecnología se integre de forma efectiva en las aulas, los docentes necesitan un profundo conocimiento del contenido, sobre la enseñanza y el aprendizaje (CDC) y sobre la tecnología (Niess, 2005). Mishra y Koehler (2006) fueron los primeros en definir claramente el modelo y las interrelaciones entre los tres dominios, incluidas las intersecciones que se producen entre ellos. Las definiciones de cada uno de los dominios del modelo fueron abundantemente tratadas y sintetizadas en innumerables estudios, por ejemplo, en el artículo de Flores y Ortiz (2019)." \textcite{fernandoFlores2024}

\subsection{Teoría del uso de la tecnología en la educación}

Es de conocimiento universal que todas las teorías de aprendizaje tienen como común denominador, la preocupación por conocer cómo aprende el alumno para que este aprendizaje, sea significativo. Desde este punto de vista, todas las teorías (tradicionales o modernas) pueden sacar provecho de la inclusión de las TIC para mejorar el proceso de enseñanza-aprendizaje. Claro que serán mejor aprovechadas, si los educadores tienen competencias TIC que cumplan con los estándares establecidos.

La teoría constructivista pone de manifiesto al discente como eje central del proceso de enseñanza- aprendizaje, donde éste es el principal motor de su propio aprendizaje y las TIC le proveen el escenario para que esto se dé. Todas las teorías que se analizan coinciden en que el alumno puede construir su aprendizaje de forma autónoma o en colaboración e interacción con otras personas y esto puede hacerlo a través de las TIC y no depender únicamente del educador en el aula. Con el uso de las TIC el estudiante es un participante activo y autónomo a su propio ritmo. De esta forma, teorías de aprendizaje y TIC es una articulación perfecta para emprender el complejo proceso enseñanza- aprendizaje de la actualidad.

Sin embargo, es necesario recordar que la construcción del conocimiento no solo es del alumno, sino también del educador, quien deberá guiar y apoyar al desarrollo de esa construcción del conocimiento significativo. Lo bueno de este modelo es que cada centro educativo, cada profesor lo puede adaptar a sus posibilidades en recursos TIC, de acuerdo al nivel del alumno y de las competencias TIC que se tengan.

\subsubsection{Teorías de aprendizaje seleccionadas}

\paragraph{Teoría constructivista}

Basada en la construcción del conocimiento por el individuo. Su principal exponente fue Jean Piaget. Partiendo que esta teoría impulsa el aprendizaje activo donde el estudiante es el actor principal del acto educativo, son las TIC, quizás las más indicadas para ser partícipes en la construcción del conocimiento y que el alumno colabore con su propio aprendizaje. Debido a que la teoría o el enfoque constructivismo está especialmente centrada en el estudiante, éste es el actor que se lleva el Oscar. Esto exige la aplicación de diversas estrategias docentes, bajo el común denominador de que el objeto fundamental del aprendizaje escolar es la construcción del conocimiento por el alumno.

Como la filosofía constructivista busca básicamente ayudar al alumno para que se convierta él mismo en constructor de su propio conocimiento, valiéndose de la asimilación de la realidad y de la acomodación de ésta a su propia estructura mental, las TIC juegan un papel valioso al ponerle a su disposición todo un arsenal de búsqueda de información. El profesor, en este caso el software, puede actuar como facilitador en esa construcción. El educando no es sólo un procesador activo de la información. También participa como constructor de dicha información, con su interacción con el ordenador. El alumno se convierte en el motor de su propio aprendizaje, interactúa para construir conocimiento y con las TIC esta interacción se acentúa. Entonces el docente debe obtener nuevas competencias para hacer frente a este nuevo discente y a la nueva forma de cómo aprende el alumno.

Desde la pasada década se habla de un neo constructivismo. Martín Bernal (2009) destaca un nuevo constructivismo, el constructivismo tecnoeducativo, al que han llamado algunos autores, colectivismo, debido a la aparición de un espacio de encuentro efectivo y positivo entre la investigación y la práctica pedagógica y los avances tecnológicos. España cuenta con defensores de este nuevo constructivismo tecnoeducativo, el modelo pedagógico CAIT (aprendizaje constructivo, auto-regulado, interactivo, y tecnológico) promovido por el profesor Jesús Beltrán (Martín, Beltrán \& Pérez, 2004) que representa la secuencia del aprendizaje, así entendido, en cinco grandes procesos, sensibilización, elaboración, personalización, aplicación y evaluación. Según este modelo el primer proceso, la sensibilización, constituye el contexto mental que el alumno necesita para aprender significativamente.

Benito (2009) señala que el constructivismo converge y se asocia desde un principio con la Red, porque ésta es un universo con el que comparte un nexo importante: ambos representan la innovación.

Además de las teorías tradicionales (asociacionistas, cognitivas, estructuralista) debido a la inclusión de las TIC en el quehacer pedagógico y el cambio que implica en la didáctica, ha dado lugar a la teoría computacional. Aunque algunos autores consideran que el sistema cognitivo del ser humano forma parte de un organismo complejo, que no puede reducirse a un mero mecanismo como ocurre con las computadoras.

\paragraph{Teoría computacional}

También llamada la teoría del procesamiento de la información. Concibe la mente humana como una computadora, donde se procesa la información adquirida. Se considera a Robert Gagné como su gestor.

Está centrada en las teorías de origen psicológico, aquellas que se aplican a la adquisición de significados por un sistema de procesamiento. Busca la adquisición de significados por un sistema de procesamiento, donde el sujeto no es un ente pasivo, sino activo, los estados mentales tienen intencionalidad para construir. El aprendizaje es concebido como un proceso que reestructura el conocimiento ya adquirido. Ocurre lo que considero un proceso de autorregulación del aprendizaje al contrastar lo adquirido con lo nuevo.

La introducción de las nuevas tecnologías denominadas por algunos NTIC han cambiado la enseñanza, existen nuevos soportes de la educación moderna como los software educativos y la Internet, la plataforma multimedia que han revolucionado el sistema educativo tradicional.

\paragraph{Pedagogía de la información}

La Teoría de aprendizaje en la pedagogía de la información pone de manifiesto que la sociedad actual, la sociedad del conocimiento o del aprendizaje y la futura, focaliza su sistema educativo en forma muy diferente a épocas pasadas. Situación esta que es de esperarse debido a los cambios tecnológicos de hoy día. La educación del siglo XXI está mediada por las nuevas tecnologías de la información y la comunicación (Meléndez, 2013).

Es recalcable que para esta sociedad lo más importante es la información y el conocimiento. La pedagogía de la información por su propia conceptualización, como ya se ha expuesto, está íntimamente relacionada con las TIC. Éstas nos permiten accesar a la información más reciente e incluso comunicarse con los autores, además de las fuentes secundarias y a los trabajos menos recientes e históricos, a los cuales en muchos casos no sería posible acceder.

\paragraph{La teoría de acción comunicativa}

La teoría de acción comunicativa propuesta por Jürgen Habermas, está basada en la relación comunicacional lingüística. La estructura de este modelo se ha utilizado para describir el modelo de comunicación que se establece a través de las páginas Web. Sustentada en el rigor, la racionalidad y la crítica, impulsando cierta capacidad de expresarse, hacerse entender y actuar coherentemente, también es congruente con las aristas de la telemática y sus recursos lógicos (Jordi, 2007).

Como se ha expresado vemos que se sustenta en la comunicación, que es una acción social y que se desarrolla muy bien con la Internet y sus diferentes aplicaciones sociales. Hoy más que nunca la comunicación no tiene barreras de tiempo, puede darse de forma síncrona y asíncrona, con lo cual el estudiante puede estar siempre en comunicación con el docente o con los compañeros.

La popularización y masificación en el caso de las redes sociales en sus diferentes vertientes y aplicaciones es una causal de la teoría de acción de la comunicación a través de las TIC. Por ende, es una teoría que bien aplicada de forma pedagógica y didáctica en el aula debe rendir resultados de aprendizaje valiosos.

\paragraph{Teoría del conocimiento situado}

Propuesta por Young (1993). Internet responde a las premisas del conocimiento situado en dos de sus características: realismo y complejidad. La Web posibilita la comunicación, el intercambio e interacción entre los usuarios que comparten afinidades de intereses.

Según esta teoría la Internet es un medio de aprendizaje, porque propicia innovadores entornos. Concibe el conocimiento como una relación activa entre un sujeto y el entorno, por lo que el aprendizaje se da cuando el alumno se involucra en forma activa en un contexto complejo y realístico, como lo es la Internet. Además, posibilita la integración y desarrollo del conocimiento al situar la Internet como un repositorio de conocimiento que bien planificado y organizado los aprendizajes a través de la Internet, proporcionan el descubrimiento y adquisición de saberes por parte del estudiante.

Sin duda, la Internet es la herramienta más utilizada y eficaz en el aprendizaje de hoy, por lo tanto su uso es ilimitado en cuanto a las diferentes formas en que el alumno concibe su aprendizaje.

\paragraph{modelo conversacional colaborativo}

Propuesto por Martin, García \& Ramírez (2006). Surge a raíz del nuevo paradigma del e-learning. Este modelo mantiene la estructura de Laurillard (1993), pero incluye en el mismo, al grupo como nuevo actor y conversador. Al incluir al grupo facilita que la Internet participe de este tipo de aprendizaje porque a través de esta herramienta se permite la conversación entre grupos, a través de las redes o grupos específicos, donde se pueden dar forum de discusión o la tradicional lluvia de ideas e intercambio de documentación entre otros.

Este modelo es comparable con el modelo de la teoría de acción comunicativa. En ambos casos, los grupos pueden comunicarse a través de la Internet para trabajos colaborativos, grupales, aclarar dudas o compartir. Máxime que muchos programas informáticos (Skype, Ustream producer, Youtube) entre otros permiten conversar con la persona que está detrás de la pantalla, con la ventaja de poder verse y compartir actividades. Incluso permiten grupos de participantes en la conversación, facilitando la conversación grupal entre diferentes miembros de un colectivo.

Se puede decir que la aplicación estrella para trabajos colaborativos, en cuanto a consultas, acuerdos e información casi síncrona, son los chat de aplicaciones como WhatsApp o Telegram, por señalar las de mayor incidencia en nuestro contexto americano.

\subsubsection{Análisis de las Teoría del uso de la tecnología en la educación}

El estudio "Evaluación Crítica del Uso de la Tecnología en la Educación" de \textcite{viera2025evaluacion} proporciona una base teórica para comprender la integración de herramientas tecnológicas en los procesos educativos. La premisa fundamental es que la tecnología no es una "solución mágica" ni una panacea para los desafíos educativos; su efectividad depende críticamente de cómo se implementa y del contexto en el que se utiliza \textcite{viera2025evaluacion}.
La investigación se enmarca en un paradigma humanista y utiliza una revisión de literatura para deconstruir varios mitos persistentes en el ámbito de la tecnología educativa. \textcite{viera2025evaluacion} concluye que es necesario adoptar un "enfoque equilibrado", reconociendo tanto las ventajas como las limitaciones de la tecnología.

Uno de los pilares de esta teoría es que el impacto real de la tecnología depende de su adecuada integración con enfoques pedagógicos efectivos. El simple uso de herramientas tecnológicas no garantiza automáticamente una mejora en el aprendizaje. Por el contrario, su efectividad está condicionada a una implementación reflexiva y contextualizada, alineada con objetivos pedagógicos claros \textcite{viera2025evaluacion}.
Además, \textcite{viera2025evaluacion} aborda mitos comunes y ofrece una perspectiva crítica sobre ellos:
La tecnología como reemplazo de la educación tradicional: El autor sostiene que la tecnología debe ser vista como una herramienta para mejorar y extender el proceso de aprendizaje humano, no como un sustituto completo.
Mejora automática del aprendizaje: Se argumenta que la tecnología puede mejorar aspectos del proceso educativo, como la personalización y la motivación, pero no garantiza por sí misma un aprendizaje más duradero o efectivo \textcite{viera2025evaluacion}.

Igualdad de oportunidades: \textcite{viera2025evaluacion} señala que la brecha digital es una barrera importante, especialmente en América Latina. La implementación desigual de la tecnología puede, de hecho, ampliar las desigualdades educativas existentes en lugar de reducirlas.
Innovación inherente: La verdadera innovación no radica en el simple uso de la tecnología, sino en la habilidad de emplearla de manera efectiva y significativa para satisfacer las necesidades de los estudiantes y los objetivos educativos \textcite{viera2025evaluacion}.

La conclusión central del trabajo de \textcite{viera2025evaluacion} es que el éxito de las tecnologías educativas se basa en una implementación adecuada y en su integración con estrategias pedagógicas sólidas. Factores como la formación docente continua, el acceso equitativo a los recursos y la preparación técnica de los educadores son cruciales para maximizar el impacto positivo de la tecnología en la educación. Por lo tanto, se requiere un uso reflexivo y basado en la evidencia, evitando expectativas poco realistas sobre la eficacia de la tecnología por sí sola \cite{viera2025evaluacion}.

