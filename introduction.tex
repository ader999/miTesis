\section{Introducción}
\label{sec:introduccion}

\subsection{Antecedentes del Problema}

En esta subsección se describen los antecedentes del problema de investigación. Aquí se proporciona un contexto histórico y teórico que fundamenta el estudio actual.

\subsection{Contexto de la Investigación}

Se presenta el contexto en el que se desarrolla la investigación, incluyendo el entorno académico, social o profesional relevante.

\subsection{Objetivos}

\subsubsection{Objetivo General}

El objetivo general de la investigación es [describir el objetivo principal].

\subsubsection{Objetivos Específicos}

Los objetivos específicos incluyen:

- Objetivo específico 1: [descripción].

- Objetivo específico 2: [descripción].

- Objetivo específico 3: [descripción].

\subsection{Preguntas Centrales de la Investigación}

Las preguntas centrales que guían este estudio son:

- ¿Pregunta 1?

- ¿Pregunta 2?

- ¿Pregunta 3?

\subsection{Justificación}

Se explica la importancia y justificación del estudio, destacando su contribución al campo de conocimiento y su relevancia práctica.

\subsection{Limitaciones del Estudio}

Se identifican las limitaciones del estudio, incluyendo restricciones metodológicas, de alcance o de recursos.

\subsection{Supuestos Básicos}

Los supuestos básicos en los que se basa la investigación son [describir los supuestos].

\subsection{Hipótesis}

Las hipótesis planteadas para este estudio son:

- Hipótesis 1: [descripción].

- Hipótesis 2: [descripción].

\subsection{Análisis de Variables}

Se describe el análisis de las variables involucradas en la investigación, incluyendo variables dependientes, independientes y de control.

\subsection{Categorías Tema Patrones Emergentes de la Investigación}

Se presentan las categorías temáticas y los patrones emergentes identificados durante el proceso de investigación.