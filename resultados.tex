% !TEX root = main.tex
\label{sec:resultados}

\subsection{Primera encuesta aplicada “Evaluación del Proceso de Planeación Académica en la Universidad Martín Lutero Sede Jalapa”}

\paragraph{1. ¿Cuánto tiempo promedio dedica actualmente a la elaboración de un plan de estudio?}

\begin{table}[H]
    \centering
    \caption{Tiempo promedio dedicado a la elaboración de un plan de estudio}
    \label{tab:p1_tiempo_dedicado}
    \begin{tabular}{lc}
        \toprule
        \textbf{Tiempo dedicado} & \textbf{Porcentaje} \\
        \midrule
        Menos de 2 horas & 20\% \\
        Entre 2 y 5 horas & 70\% \\
        Más de 5 horas & 10\% \\
        \bottomrule
    \end{tabular}
\end{table}

\begin{figure}[H]
    \centering
    % \includegraphics[width=0.8\textwidth]{figuras/primera_encuesta/figura4.png} % Insertar imagen aquí
    \caption{Tiempo dedicado a la planeación}
    \label{fig:p1_tiempo_dedicado}
\end{figure}

La mayoría de los docentes (70\%) indicaron que dedican entre 2 y 5 horas a la planificación de sus clases, lo que sugiere que este proceso requiere una inversión de tiempo considerable. Solo el 20\% de los encuestados afirmó que puede completar su planificación en menos de 2 horas, mientras que un 10\% reportó que necesita más de 5 horas para realizar esta tarea.

\paragraph{2. ¿Qué herramientas utiliza principalmente para crear los planes de estudio y sílabos?}

\begin{table}[H]
    \centering
    \caption{Herramientas utilizadas para crear planes de estudio}
    \label{tab:p2_herramientas}
    \begin{tabular}{lc}
        \toprule
        \textbf{Herramientas} & \textbf{Porcentaje} \\
        \midrule
        Word o Excel & 80\% \\
        Sistema manual en Papel & 10\% \\
        Virtual & 10\% \\
        \bottomrule
    \end{tabular}
\end{table}

\begin{figure}[H]
    \centering
    % \includegraphics[width=0.8\textwidth]{figuras/primera_encuesta/figura5.png} % Insertar imagen aquí
    \caption{Herramientas principales de planeación}
    \label{fig:p2_herramientas}
\end{figure}

Los datos muestran que la gran mayoría de los docentes (80\%) prefieren el uso de Microsoft Word o Excel para la elaboración de sus planes de estudio y sílabos. Esto indica que los docentes ya están familiarizados con herramientas digitales, aunque estas no están específicamente diseñadas para la planificación académica. Un 10\% de los encuestados sigue utilizando un sistema manual en papel, lo que puede representar dificultades en términos de almacenamiento, edición y recuperación de información. Finalmente, otro 10\% utiliza una herramienta virtual, lo que sugiere que existe cierta apertura hacia soluciones digitales especializadas. Estos resultados evidencian la necesidad de un sistema más eficiente y adaptado a la planificación académica, como PLANEAUML, que podría automatizar tareas, mejorar la organización de la información y reducir el tiempo invertido en la planificación.

\paragraph{3. ¿Qué tan fácil considera cumplir con los plazos establecidos para la entrega de sílabos?}

\begin{table}[H]
    \centering
    \caption{Facilidad para cumplir plazos de entrega}
    \label{tab:p3_plazos}
    \begin{tabular}{lc}
        \toprule
        \textbf{Facilidad de entrega} & \textbf{Porcentaje} \\
        \midrule
        Muy Fácil & 20\% \\
        Fácil & 40\% \\
        Difícil & 30\% \\
        Muy Difícil & 10\% \\
        \bottomrule
    \end{tabular}
\end{table}

\begin{figure}[H]
    \centering
    % \includegraphics[width=0.8\textwidth]{figuras/primera_encuesta/figura6.png} % Insertar imagen aquí
    \caption{Percepción sobre el cumplimiento de plazos}
    \label{fig:p3_plazos}
\end{figure}

Los resultados indican que el 60\% de los docentes perciben que cumplir con los plazos establecidos es relativamente sencillo, mientras que un 40\% encuentra dificultades. Esto sugiere que, si bien la mayoría logra adaptarse, existe un segmento significativo que enfrenta obstáculos, posiblemente por la carga laboral o la falta de herramientas optimizadas. La digitalización y automatización de los procesos académicos podría reducir estas dificultades al proporcionar flujos de trabajo más eficientes y organizados.

\paragraph{4. ¿Con qué frecuencia enfrenta problemas como errores o retrabajo en la planeación académica?}

\begin{table}[H]
    \centering
    \caption{Frecuencia de errores o retrabajo}
    \label{tab:p4_errores}
    \begin{tabular}{lc}
        \toprule
        \textbf{Frecuencia de errores} & \textbf{Porcentaje} \\
        \midrule
        Nunca & 10\% \\
        Ocasionalmente & 50\% \\
        Frecuentemente & 30\% \\
        Siempre & 10\% \\
        \bottomrule
    \end{tabular}
\end{table}

\begin{figure}[H]
    \centering
    % \includegraphics[width=0.8\textwidth]{figuras/primera_encuesta/figura7.png} % Insertar imagen aquí
    \caption{Frecuencia de problemas en la planeación}
    \label{fig:p4_errores}
\end{figure}

Un 40\% de los encuestados reporta problemas frecuentes en la planeación académica, lo que sugiere que las herramientas actuales no son lo suficientemente eficientes. Gran parte de esta dificultad podría atribuirse a la complejidad de gestionar datos estructurados en hojas de cálculo como Excel, así como a la falta de sistemas digitales verdaderamente integrados y de capacitación en su uso.

\paragraph{5. ¿Cuáles son los principales problemas que enfrenta durante la planeación académica?}

\begin{table}[H]
    \centering
    \caption{Principales problemas en la planeación académica}
    \label{tab:p5_problemas}
    \begin{tabular}{p{10cm}c}
        \toprule
        \textbf{Principales problemas} & \textbf{Porcentaje} \\
        \midrule
        Falta de tiempo & 50\% \\
        Errores frecuentes o retrabajo & 0\% \\
        Dificultades técnicas con las herramientas actuales & 40\% \\
        Adecuar mi plan didáctico según los parámetros de la universidad & 10\% \\
        \bottomrule
    \end{tabular}
\end{table}

\begin{figure}[H]
    \centering
    % \includegraphics[width=0.8\textwidth]{figuras/primera_encuesta/figura8.png} % Insertar imagen aquí
    \caption{Problemas principales enfrentados}
    \label{fig:p5_problemas}
\end{figure}

La falta de tiempo (50\%) y las dificultades técnicas con las herramientas actuales (40\%) son los problemas más importantes en la planificación académica. Esto demuestra que las herramientas disponibles no son eficientes y dificultan el trabajo docente. La implementación de plataformas digitales especializadas podría ser una solución para reducir la carga de trabajo y optimizar la estructuración de los planes académicos.

\paragraph{6. ¿Cuánto tiempo promedio invierte en corregir errores o ajustes en los planes de estudio ya elaborados?}

\begin{table}[H]
    \centering
    \caption{Tiempo invertido en correcciones}
    \label{tab:p6_correcciones}
    \begin{tabular}{lc}
        \toprule
        \textbf{Tiempo en correcciones} & \textbf{Porcentaje} \\
        \midrule
        Menos de 1 Hora & 50\% \\
        Entre 1 y 3 Horas & 20\% \\
        Más de 3 Horas & 20\% \\
        No realizo Correcciones & 10\% \\
        \bottomrule
    \end{tabular}
\end{table}

\begin{figure}[H]
    \centering
    % \includegraphics[width=0.8\textwidth]{figuras/primera_encuesta/figura9.png} % Insertar imagen aquí
    \caption{Tiempo dedicado a correcciones}
    \label{fig:p6_correcciones}
\end{figure}

El 50\% de los docentes dedica menos de una hora a correcciones, lo que indica que los errores no son excesivamente graves. Sin embargo, el 40\% necesita más de una hora para realizar ajustes, lo que resalta la necesidad de un sistema más eficiente para minimizar el tiempo dedicado a correcciones.

\paragraph{7. ¿Qué aspectos del proceso de planeación académica le resultan más complicados?}

\begin{table}[H]
    \centering
    \caption{Aspectos más complicados de la planeación}
    \label{tab:p7_complicaciones}
    \begin{tabular}{p{10cm}c}
        \toprule
        \textbf{Aspectos complicados} & \textbf{Porcentaje} \\
        \midrule
        Identificar los contenidos clave para cada asignatura & 8.33\% \\
        Ajustarse a las directrices institucionales & 25\% \\
        Cumplir con los plazos de entrega & 41.67\% \\
        Otros & 25\% \\
        \bottomrule
    \end{tabular}
\end{table}

\begin{figure}[H]
    \centering
    % \includegraphics[width=0.8\textwidth]{figuras/primera_encuesta/figura10.png} % Insertar imagen aquí
    \caption{Aspectos complicados del proceso}
    \label{fig:p7_complicaciones}
\end{figure}

El 42\% de los docentes considera que cumplir con los plazos de entrega es el aspecto más complicado del proceso de planificación académica, seguido por la necesidad de ajustarse a las directrices institucionales (25\%). Esto sugiere que, aunque los docentes cuentan con un marco regulador, la rigidez de estos lineamientos y la falta de flexibilidad pueden generar desafíos adicionales.

\paragraph{8. ¿Qué tan satisfecho está con las herramientas que utiliza actualmente para elaborar los sílabos y generar los planes de clase?}

\begin{table}[H]
    \centering
    \caption{Nivel de satisfacción con herramientas actuales}
    \label{tab:p8_satisfaccion}
    \begin{tabular}{lc}
        \toprule
        \textbf{Nivel de Satisfacción} & \textbf{Porcentaje} \\
        \midrule
        Muy Satisfecho & 20\% \\
        Satisfecho & 40\% \\
        Insatisfecho & 30\% \\
        Muy Insatisfecho & 10\% \\
        \bottomrule
    \end{tabular}
\end{table}

\begin{figure}[H]
    \centering
    % \includegraphics[width=0.8\textwidth]{figuras/primera_encuesta/figura11.png} % Insertar imagen aquí
    \caption{Satisfacción con las herramientas actuales}
    \label{fig:p8_satisfaccion}
\end{figure}

El nivel de satisfacción de los docentes con las herramientas actuales muestra que el 40\% está satisfecho, pero un 40\% está insatisfecho o muy insatisfecho. Esto indica que, si bien algunas herramientas cumplen con las expectativas, una proporción significativa de los docentes considera que hay margen de mejora.

\paragraph{9. ¿Cómo calificaría la coordinación con otros docentes o autoridades durante la planeación académica?}

\begin{table}[H]
    \centering
    \caption{Calificación de la coordinación académica}
    \label{tab:p9_coordinacion}
    \begin{tabular}{lc}
        \toprule
        \textbf{Coordinación} & \textbf{Porcentaje} \\
        \midrule
        Muy Buena & 40\% \\
        Buena & 40\% \\
        Regular & 10\% \\
        Mala & 10\% \\
        \bottomrule
    \end{tabular}
\end{table}

\begin{figure}[H]
    \centering
    % \includegraphics[width=0.8\textwidth]{figuras/primera_encuesta/figura12.png} % Insertar imagen aquí
    \caption{Percepción de la coordinación}
    \label{fig:p9_coordinacion}
\end{figure}

El 80\% de los docentes considera que la coordinación con sus colegas y autoridades es buena o muy buena, lo que sugiere un entorno colaborativo en la planificación académica. Sin embargo, un 20\% la califica como regular o mala, lo que indica que aún existen áreas de mejora en la comunicación y cooperación institucional.

\paragraph{10. ¿Qué tan útil considera recibir capacitaciones periódicas sobre nuevas herramientas o estrategias de planeación académica?}

\begin{table}[H]
    \centering
    \caption{Utilidad de capacitaciones periódicas}
    \label{tab:p10_capacitaciones}
    \begin{tabular}{lc}
        \toprule
        \textbf{Utilidad de las Capacitaciones} & \textbf{Porcentaje} \\
        \midrule
        Muy Útil & 80\% \\
        Útil & 0\% \\
        Poco Útil & 20\% \\
        Nada Útil & 0\% \\
        \bottomrule
    \end{tabular}
\end{table}

\begin{figure}[H]
    \centering
    % \includegraphics[width=0.8\textwidth]{figuras/primera_encuesta/figura13.png} % Insertar imagen aquí
    \caption{Interés en capacitaciones}
    \label{fig:p10_capacitaciones}
\end{figure}

La gran mayoría de los encuestados (80\%) considera que recibir capacitaciones periódicas es muy útil, lo que resalta la necesidad de programas de formación continua en herramientas digitales y estrategias pedagógicas.

\paragraph{11. ¿Con qué frecuencia utiliza recursos externos (como guías, manuales o ejemplos) para elaborar sílabos?}

\begin{table}[H]
    \centering
    \caption{Uso de recursos externos}
    \label{tab:p11_recursos}
    \begin{tabular}{lc}
        \toprule
        \textbf{Recursos Externos} & \textbf{Porcentaje} \\
        \midrule
        Siempre & 50\% \\
        Frecuentemente & 40\% \\
        Ocasionalmente & 0\% \\
        Nunca & 10\% \\
        \bottomrule
    \end{tabular}
\end{table}

\begin{figure}[H]
    \centering
    % \includegraphics[width=0.8\textwidth]{figuras/primera_encuesta/figura14.png} % Insertar imagen aquí
    \caption{Frecuencia de uso de recursos externos}
    \label{fig:p11_recursos}
\end{figure}

El 90\% de los docentes afirma utilizar recursos externos siempre o con frecuencia, lo que indica que la consulta de material complementario es una práctica común en la planificación académica. Este hallazgo sugiere que disponer de una biblioteca digital o repositorio institucional con materiales de referencia podría facilitar aún más este proceso.

\paragraph{12. ¿Le gustaría contar con un sistema automatizado que simplifique la planeación académica?}

\begin{table}[H]
    \centering
    \caption{Interés en un sistema automatizado}
    \label{tab:p12_sistema}
    \begin{tabular}{lc}
        \toprule
        \textbf{Sistema} & \textbf{Porcentaje} \\
        \midrule
        Sí, Definitivamente & 80\% \\
        Podría ser Útil & 10\% \\
        No estoy interesado & 10\% \\
        \bottomrule
    \end{tabular}
\end{table}

\begin{figure}[H]
    \centering
    % \includegraphics[width=0.8\textwidth]{figuras/primera_encuesta/figura15.png} % Insertar imagen aquí
    \caption{Aceptación de un sistema automatizado}
    \label{fig:p12_sistema}
\end{figure}

El 80\% de los docentes expresó un claro interés en la implementación de un sistema automatizado para la planeación académica, lo que refleja la necesidad de optimizar este proceso. Un 10\% considera que podría ser útil, mientras que solo un 10\% no está interesado.

\subsection{Segunda encuesta aplicada “Evaluación de la aplicación web PLANEAUML”}

\subsubsection{Resultado de las preguntas para Administrativos de la Universidad MARTÍN LUTERO Sede Jalapa}

\subsubsection{Resultado de las preguntas para Administrativos de la Universidad MARTÍN LUTERO Sede Jalapa.}

\subsection{Tercera encuesta aplicada seis meses después de la implementación de PLANEAUML}
