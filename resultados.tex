% !TEX root = main.tex
\label{sec:resultados}

\subsection{Primera encuesta aplicada “Evaluación del Proceso de Planeación Académica en la Universidad Martín Lutero Sede Jalapa”}

\paragraph{1. ¿Cuánto tiempo promedio dedica actualmente a la elaboración de un plan de estudio?}

\begin{table}[H]
    \centering
    \caption{Tiempo promedio dedicado a la elaboración de un plan de estudio}
    \label{tab:p1_tiempo_dedicado}
    \begin{tabular}{lc}
        \toprule
        \textbf{Tiempo dedicado} & \textbf{Porcentaje} \\
        \midrule
        Menos de 2 horas & 20\% \\
        Entre 2 y 5 horas & 70\% \\
        Más de 5 horas & 10\% \\
        \bottomrule
    \end{tabular}
\end{table}

\begin{figure}[H]
    \centering
     \includegraphics[width=0.8\textwidth]{figuras/primera_encuesta/encuesta1.png}
    \caption{Tiempo dedicado a la planeación}
    \label{fig:p1_tiempo_dedicado}
\end{figure}

La mayoría de los docentes (70\%) indicaron que dedican entre 2 y 5 horas a la planificación de sus clases, lo que sugiere que este proceso requiere una inversión de tiempo considerable. Solo el 20\% de los encuestados afirmó que puede completar su planificación en menos de 2 horas, mientras que un 10\% reportó que necesita más de 5 horas para realizar esta tarea.

\paragraph{2. ¿Qué herramientas utiliza principalmente para crear los planes de estudio y sílabos?}

\begin{table}[H]
    \centering
    \caption{Herramientas utilizadas para crear planes de estudio}
    \label{tab:p2_herramientas}
    \begin{tabular}{lc}
        \toprule
        \textbf{Herramientas} & \textbf{Porcentaje} \\
        \midrule
        Word o Excel & 80\% \\
        Sistema manual en Papel & 10\% \\
        Virtual & 10\% \\
        \bottomrule
    \end{tabular}
\end{table}

\begin{figure}[H]
    \centering
    \includegraphics[width=0.8\textwidth]{figuras/primera_encuesta/encuesta2.png} % 
    \caption{Herramientas principales de planeación}
    \label{fig:p2_herramientas}
\end{figure}

Los datos muestran que la gran mayoría de los docentes (80\%) prefieren el uso de Microsoft Word o Excel para la elaboración de sus planes de estudio y sílabos. Esto indica que los docentes ya están familiarizados con herramientas digitales, aunque estas no están específicamente diseñadas para la planificación académica. Un 10\% de los encuestados sigue utilizando un sistema manual en papel, lo que puede representar dificultades en términos de almacenamiento, edición y recuperación de información. Finalmente, otro 10\% utiliza una herramienta virtual, lo que sugiere que existe cierta apertura hacia soluciones digitales especializadas. Estos resultados evidencian la necesidad de un sistema más eficiente y adaptado a la planificación académica, como PLANEAUML, que podría automatizar tareas, mejorar la organización de la información y reducir el tiempo invertido en la planificación.

\paragraph{3. ¿Qué tan fácil considera cumplir con los plazos establecidos para la entrega de sílabos?}

\begin{table}[H]
    \centering
    \caption{Facilidad para cumplir plazos de entrega}
    \label{tab:p3_plazos}
    \begin{tabular}{lc}
        \toprule
        \textbf{Facilidad de entrega} & \textbf{Porcentaje} \\
        \midrule
        Muy Fácil & 20\% \\
        Fácil & 40\% \\
        Difícil & 30\% \\
        Muy Difícil & 10\% \\
        \bottomrule
    \end{tabular}
\end{table}

\begin{figure}[H]
    \centering
    \includegraphics[width=0.8\textwidth]{figuras/primera_encuesta/encuesta3.png} 
    \caption{Percepción sobre el cumplimiento de plazos}
    \label{fig:p3_plazos}
\end{figure}

Los resultados indican que el 60\% de los docentes perciben que cumplir con los plazos establecidos es relativamente sencillo, mientras que un 40\% encuentra dificultades. Esto sugiere que, si bien la mayoría logra adaptarse, existe un segmento significativo que enfrenta obstáculos, posiblemente por la carga laboral o la falta de herramientas optimizadas. La digitalización y automatización de los procesos académicos podría reducir estas dificultades al proporcionar flujos de trabajo más eficientes y organizados.

\paragraph{4. ¿Con qué frecuencia enfrenta problemas como errores o retrabajo en la planeación académica?}

\begin{table}[H]
    \centering
    \caption{Frecuencia de errores o retrabajo}
    \label{tab:p4_errores}
    \begin{tabular}{lc}
        \toprule
        \textbf{Frecuencia de errores} & \textbf{Porcentaje} \\
        \midrule
        Nunca & 10\% \\
        Ocasionalmente & 50\% \\
        Frecuentemente & 30\% \\
        Siempre & 10\% \\
        \bottomrule
    \end{tabular}
\end{table}

\begin{figure}[H]
    \centering
    \includegraphics[width=0.8\textwidth]{figuras/primera_encuesta/encuesta4.png} 
    \caption{Frecuencia de problemas en la planeación}
    \label{fig:p4_errores}
\end{figure}

Un 40\% de los encuestados reporta problemas frecuentes en la planeación académica, lo que sugiere que las herramientas actuales no son lo suficientemente eficientes. Gran parte de esta dificultad podría atribuirse a la complejidad de gestionar datos estructurados en hojas de cálculo como Excel, así como a la falta de sistemas digitales verdaderamente integrados y de capacitación en su uso.

\paragraph{5. ¿Cuáles son los principales problemas que enfrenta durante la planeación académica?}

\begin{table}[H]
    \centering
    \caption{Principales problemas en la planeación académica}
    \label{tab:p5_problemas}
    \begin{tabular}{p{10cm}c}
        \toprule
        \textbf{Principales problemas} & \textbf{Porcentaje} \\
        \midrule
        Falta de tiempo & 50\% \\
        Errores frecuentes o retrabajo & 0\% \\
        Dificultades técnicas con las herramientas actuales & 40\% \\
        Adecuar mi plan didáctico según los parámetros de la universidad & 10\% \\
        \bottomrule
    \end{tabular}
\end{table}

\begin{figure}[H]
    \centering
    \includegraphics[width=0.8\textwidth]{figuras/primera_encuesta/encuesta5.png} 
    \caption{Problemas principales enfrentados}
    \label{fig:p5_problemas}
\end{figure}

La falta de tiempo (50\%) y las dificultades técnicas con las herramientas actuales (40\%) son los problemas más importantes en la planificación académica. Esto demuestra que las herramientas disponibles no son eficientes y dificultan el trabajo docente. La implementación de plataformas digitales especializadas podría ser una solución para reducir la carga de trabajo y optimizar la estructuración de los planes académicos.

\paragraph{6. ¿Cuánto tiempo promedio invierte en corregir errores o ajustes en los planes de estudio ya elaborados?}

\begin{table}[H]
    \centering
    \caption{Tiempo invertido en correcciones}
    \label{tab:p6_correcciones}
    \begin{tabular}{lc}
        \toprule
        \textbf{Tiempo en correcciones} & \textbf{Porcentaje} \\
        \midrule
        Menos de 1 Hora & 50\% \\
        Entre 1 y 3 Horas & 20\% \\
        Más de 3 Horas & 20\% \\
        No realizo Correcciones & 10\% \\
        \bottomrule
    \end{tabular}
\end{table}

\begin{figure}[H]
    \centering
    \includegraphics[width=0.8\textwidth]{figuras/primera_encuesta/encuesta6.png} 
    \caption{Tiempo dedicado a correcciones}
    \label{fig:p6_correcciones}
\end{figure}

El 50\% de los docentes dedica menos de una hora a correcciones, lo que indica que los errores no son excesivamente graves. Sin embargo, el 40\% necesita más de una hora para realizar ajustes, lo que resalta la necesidad de un sistema más eficiente para minimizar el tiempo dedicado a correcciones.

\paragraph{7. ¿Qué aspectos del proceso de planeación académica le resultan más complicados?}

\begin{table}[H]
    \centering
    \caption{Aspectos más complicados de la planeación}
    \label{tab:p7_complicaciones}
    \begin{tabular}{p{10cm}c}
        \toprule
        \textbf{Aspectos complicados} & \textbf{Porcentaje} \\
        \midrule
        Identificar los contenidos clave para cada asignatura & 8.33\% \\
        Ajustarse a las directrices institucionales & 25\% \\
        Cumplir con los plazos de entrega & 41.67\% \\
        Otros & 25\% \\
        \bottomrule
    \end{tabular}
\end{table}

\begin{figure}[H]
    \centering
     \includegraphics[width=0.8\textwidth]{figuras/primera_encuesta/encuesta7.png}
    \caption{Aspectos complicados del proceso}
    \label{fig:p7_complicaciones}
\end{figure}

El 42\% de los docentes considera que cumplir con los plazos de entrega es el aspecto más complicado del proceso de planificación académica, seguido por la necesidad de ajustarse a las directrices institucionales (25\%). Esto sugiere que, aunque los docentes cuentan con un marco regulador, la rigidez de estos lineamientos y la falta de flexibilidad pueden generar desafíos adicionales.

\paragraph{8. ¿Qué tan satisfecho está con las herramientas que utiliza actualmente para elaborar los sílabos y generar los planes de clase?}

\begin{table}[H]
    \centering
    \caption{Nivel de satisfacción con herramientas actuales}
    \label{tab:p8_satisfaccion}
    \begin{tabular}{lc}
        \toprule
        \textbf{Nivel de Satisfacción} & \textbf{Porcentaje} \\
        \midrule
        Muy Satisfecho & 20\% \\
        Satisfecho & 40\% \\
        Insatisfecho & 30\% \\
        Muy Insatisfecho & 10\% \\
        \bottomrule
    \end{tabular}
\end{table}

\begin{figure}[H]
    \centering
    \includegraphics[width=0.8\textwidth]{figuras/primera_encuesta/encuesta8.png} 
    \caption{Satisfacción con las herramientas actuales}
    \label{fig:p8_satisfaccion}
\end{figure}

El nivel de satisfacción de los docentes con las herramientas actuales muestra que el 40\% está satisfecho, pero un 40\% está insatisfecho o muy insatisfecho. Esto indica que, si bien algunas herramientas cumplen con las expectativas, una proporción significativa de los docentes considera que hay margen de mejora.

\paragraph{9. ¿Cómo calificaría la coordinación con otros docentes o autoridades durante la planeación académica?}

\begin{table}[H]
    \centering
    \caption{Calificación de la coordinación académica}
    \label{tab:p9_coordinacion}
    \begin{tabular}{lc}
        \toprule
        \textbf{Coordinación} & \textbf{Porcentaje} \\
        \midrule
        Muy Buena & 40\% \\
        Buena & 40\% \\
        Regular & 10\% \\
        Mala & 10\% \\
        \bottomrule
    \end{tabular}
\end{table}

\begin{figure}[H]
    \centering
    \includegraphics[width=0.8\textwidth]{figuras/primera_encuesta/encuesta9.png}
    \caption{Percepción de la coordinación}
    \label{fig:p9_coordinacion}
\end{figure}

El 80\% de los docentes considera que la coordinación con sus colegas y autoridades es buena o muy buena, lo que sugiere un entorno colaborativo en la planificación académica. Sin embargo, un 20\% la califica como regular o mala, lo que indica que aún existen áreas de mejora en la comunicación y cooperación institucional.

\paragraph{10. ¿Qué tan útil considera recibir capacitaciones periódicas sobre nuevas herramientas o estrategias de planeación académica?}

\begin{table}[H]
    \centering
    \caption{Utilidad de capacitaciones periódicas}
    \label{tab:p10_capacitaciones}
    \begin{tabular}{lc}
        \toprule
        \textbf{Utilidad de las Capacitaciones} & \textbf{Porcentaje} \\
        \midrule
        Muy Útil & 80\% \\
        Útil & 0\% \\
        Poco Útil & 20\% \\
        Nada Útil & 0\% \\
        \bottomrule
    \end{tabular}
\end{table}

\begin{figure}[H]
    \centering
    \includegraphics[width=0.8\textwidth]{figuras/primera_encuesta/encuesta10.png}
    \caption{Interés en capacitaciones}
    \label{fig:p10_capacitaciones}
\end{figure}

La gran mayoría de los encuestados (80\%) considera que recibir capacitaciones periódicas es muy útil, lo que resalta la necesidad de programas de formación continua en herramientas digitales y estrategias pedagógicas.

\paragraph{11. ¿Con qué frecuencia utiliza recursos externos (como guías, manuales o ejemplos) para elaborar sílabos?}

\begin{table}[H]
    \centering
    \caption{Uso de recursos externos}
    \label{tab:p11_recursos}
    \begin{tabular}{lc}
        \toprule
        \textbf{Recursos Externos} & \textbf{Porcentaje} \\
        \midrule
        Siempre & 50\% \\
        Frecuentemente & 40\% \\
        Ocasionalmente & 0\% \\
        Nunca & 10\% \\
        \bottomrule
    \end{tabular}
\end{table}

\begin{figure}[H]
    \centering
    \includegraphics[width=0.8\textwidth]{figuras/primera_encuesta/encuesta11.png}
    \caption{Frecuencia de uso de recursos externos}
    \label{fig:p11_recursos}
\end{figure}

El 90\% de los docentes afirma utilizar recursos externos siempre o con frecuencia, lo que indica que la consulta de material complementario es una práctica común en la planificación académica. Este hallazgo sugiere que disponer de una biblioteca digital o repositorio institucional con materiales de referencia podría facilitar aún más este proceso.

\paragraph{12. ¿Le gustaría contar con un sistema automatizado que simplifique la planeación académica?}

\begin{table}[H]
    \centering
    \caption{Interés en un sistema automatizado}
    \label{tab:p12_sistema}
    \begin{tabular}{lc}
        \toprule
        \textbf{Sistema} & \textbf{Porcentaje} \\
        \midrule
        Sí, Definitivamente & 80\% \\
        Podría ser Útil & 10\% \\
        No estoy interesado & 10\% \\
        \bottomrule
    \end{tabular}
\end{table}

\begin{figure}[H]
    \centering
    \includegraphics[width=0.8\textwidth]{figuras/primera_encuesta/encuesta12.png}
    \caption{Aceptación de un sistema automatizado}
    \label{fig:p12_sistema}
\end{figure}

El 80\% de los docentes expresó un claro interés en la implementación de un sistema automatizado para la planeación académica, lo que refleja la necesidad de optimizar este proceso. Un 10\% considera que podría ser útil, mientras que solo un 10\% no está interesado.

\subsection{Segunda encuesta aplicada “Evaluación de la aplicación web PLANEAUML”}

\subsubsection{Resultado de las preguntas para Administrativos de la Universidad MARTÍN LUTERO Sede Jalapa}

\paragraph{1. ¿Cómo evaluaría la facilidad de uso del sistema?}

\begin{table}[H]
    \centering
    \caption{Facilidad de uso del sistema}
    \label{tab:p1_admin_facilidad}
    \begin{tabular}{lc}
        \toprule
        \textbf{Facilidad de uso} & \textbf{Porcentaje} \\
        \midrule
        Muy Fácil & 100\% \\
        Fácil & 0\% \\
        Complicado & 0\% \\
        Muy Complicado & 0\% \\
        \bottomrule
    \end{tabular}
\end{table}

\begin{figure}[H]
    \centering
    \includegraphics[width=0.8\textwidth]{figuras/segunda_encuesta/administrativos/encuesta1.png}
    \caption{Evaluación de la facilidad de uso}
    \label{fig:p1_admin_facilidad}
\end{figure}

La totalidad de los encuestados calificó la facilidad de uso del sistema como "Muy fácil". Esta percepción es crucial, ya que la facilidad de uso percibida influye directamente en la adopción y uso efectivo de sistemas tecnológicos en entornos educativos.

\paragraph{2. ¿Cree que el panel de control permite gestionar de manera eficiente a los usuarios y sus roles?}

\begin{table}[H]
    \centering
    \caption{Eficiencia del panel de control para gestión de usuarios}
    \label{tab:p2_admin_panel}
    \begin{tabular}{lc}
        \toprule
        \textbf{Panel de Control} & \textbf{Porcentaje} \\
        \midrule
        Muy Fácil & 100\% \\
        Fácil & 0\% \\
        Complicado & 0\% \\
        Muy Complicado & 0\% \\
        \bottomrule
    \end{tabular}
\end{table}

\begin{figure}[H]
    \centering
    \includegraphics[width=0.8\textwidth]{figuras/segunda_encuesta/administrativos/encuesta2.png}
    \caption{Gestión eficiente de usuarios y roles}
    \label{fig:p2_admin_panel}
\end{figure}

La totalidad de los administrativos considera que el panel de control de PLANEAUML permite una gestión eficiente de los usuarios y sus roles. Esta unanimidad sugiere que la herramienta ofrece funcionalidades adecuadas para la administración de permisos y roles dentro del sistema, facilitando la asignación de responsabilidades y el control de acceso de manera efectiva.

\paragraph{3. ¿Qué aspectos del sistema le parecen más útiles?}

\begin{table}[H]
    \centering
    \caption{Aspectos más útiles del sistema}
    \label{tab:p3_admin_utilidad}
    \begin{tabular}{p{10cm}c}
        \toprule
        \textbf{Funciones de PLANEAUML} & \textbf{Porcentaje} \\
        \midrule
        Administración y control de permiso de usuarios & 67\% \\
        Exportación de datos & 67\% \\
        Seguimiento de planes de estudios terminados & 100\% \\
        Filtros para la búsqueda de datos & 33\% \\
        \bottomrule
    \end{tabular}
\end{table}

\begin{figure}[H]
    \centering
    \includegraphics[width=0.8\textwidth]{figuras/segunda_encuesta/administrativos/encuesta3.png}
    \caption{Funcionalidades más útiles}
    \label{fig:p3_admin_utilidad}
\end{figure}

El seguimiento de planes de estudios terminados fue considerado útil por el 100\% de los encuestados, mientras que la administración de permisos y la exportación de datos fueron valoradas positivamente por el 67.7\%. Estas funcionalidades son cruciales para una gestión académica eficiente, permitiendo un control preciso y acceso adecuado a la información.

\paragraph{4. ¿Le parece eficiente la manera en que se maneja la estructura de los datos?}

\begin{table}[H]
    \centering
    \caption{Eficiencia en la estructura de datos}
    \label{tab:p4_admin_estructura}
    \begin{tabular}{lc}
        \toprule
        \textbf{Estructura de datos} & \textbf{Porcentaje} \\
        \midrule
        Sí & 100\% \\
        Tal vez & 0\% \\
        No & 0\% \\
        \bottomrule
    \end{tabular}
\end{table}

\begin{figure}[H]
    \centering
    \includegraphics[width=0.8\textwidth]{figuras/segunda_encuesta/administrativos/encuesta4.png}
    \caption{Eficiencia de la estructura de datos}
    \label{fig:p4_admin_estructura}
\end{figure}

Todos los encuestados consideraron eficiente el manejo de la estructura de los datos. Una estructura de datos bien organizada es fundamental para la eficacia de los sistemas de gestión educativa, ya que facilita el acceso y la manipulación de la información necesaria para la toma de decisiones.

\paragraph{5. ¿Le parecen eficientes los filtros aplicados en el panel de control?}

\begin{table}[H]
    \centering
    \caption{Eficiencia de los filtros del panel de control}
    \label{tab:p5_admin_filtros}
    \begin{tabular}{lc}
        \toprule
        \textbf{Eficiencia de los Filtros} & \textbf{Porcentaje} \\
        \midrule
        Sí & 68\% \\
        Tal vez & 32\% \\
        No & 0\% \\
        \bottomrule
    \end{tabular}
\end{table}

\begin{figure}[H]
    \centering
    \includegraphics[width=0.8\textwidth]{figuras/segunda_encuesta/administrativos/encuesta5.png}
    \caption{Eficiencia de los filtros}
    \label{fig:p5_admin_filtros}
\end{figure}

El 68\% de los participantes consideraron eficientes los filtros del panel de control, mientras que el 32\% respondió "Tal vez". La implementación de filtros efectivos es esencial para la usabilidad del sistema, permitiendo a los usuarios localizar rápidamente la información relevante.

\paragraph{6. ¿Cree que PLANEAUML es un buen sistema para gestionar los planes de estudio académicos?}

\begin{table}[H]
    \centering
    \caption{Evaluación general del sistema PLANEAUML}
    \label{tab:p6_admin_general}
    \begin{tabular}{lc}
        \toprule
        \textbf{Panel de Control} & \textbf{Porcentaje} \\
        \midrule
        Sí & 100\% \\
        Tal vez & 0\% \\
        No & 0\% \\
        \bottomrule
    \end{tabular}
\end{table}

\begin{figure}[H]
    \centering
    \includegraphics[width=0.8\textwidth]{figuras/segunda_encuesta/administrativos/encuesta6.png}
    \caption{Evaluación del sistema para gestión académica}
    \label{fig:p6_admin_general}
\end{figure}

La totalidad de los encuestados considera que PLANEAUML es un buen sistema para la gestión de planes de estudio académicos. La adopción de sistemas de gestión académica eficaces contribuye significativamente a la mejora de la calidad educativa y a la optimización de los procesos administrativos.

\subsubsection{Resultados de las preguntas para Docentes de la Universidad MARTÍN LUTERO Sede Jalapa}

\paragraph{1. ¿Qué le pareció el sistema PLANEAUML?}

\begin{table}[H]
    \centering
    \caption{Evaluación del sistema PLANEAUML}
    \label{tab:p1_docente_evaluacion}
    \begin{tabular}{lc}
        \toprule
        \textbf{Evaluación de PLANEAUML} & \textbf{Porcentaje} \\
        \midrule
        Bien & 94\% \\
        Regular & 6\% \\
        No satisface las necesidades & 0\% \\
        \bottomrule
    \end{tabular}
\end{table}

\begin{figure}[H]
    \centering
    \includegraphics[width=0.8\textwidth]{figuras/segunda_encuesta/docentes/encuesta1.png}
    \caption{Evaluación general del sistema}
    \label{fig:p1_docente_evaluacion}
\end{figure}

La mayoría de los docentes (94\%) calificaron el sistema como "Bien", indicando una alta satisfacción general. La percepción positiva de los docentes es crucial para la implementación exitosa de sistemas tecnológicos en entornos educativos, ya que su aceptación influye directamente en el uso efectivo de la herramienta.

\paragraph{2. ¿Qué aspectos del sistema le parecen más útiles?}

\begin{table}[H]
    \centering
    \caption{Aspectos más útiles del sistema}
    \label{tab:p2_docente_utilidad}
    \begin{tabular}{p{10cm}c}
        \toprule
        \textbf{Características del sistema} & \textbf{Porcentaje} \\
        \midrule
        Facilidad del llenado del formulario del silabo & 69\% \\
        Mejor organización y control de sus datos & 56\% \\
        Generación automática del contenido del sílabo y la Guía con (IA) & 44\% \\
        Llenado automático del Plan de clase y la Guía del estudio & 44\% \\
        Otras cualidades & 6\% \\
        \bottomrule
    \end{tabular}
\end{table}

\begin{figure}[H]
    \centering
    \includegraphics[width=0.8\textwidth]{figuras/segunda_encuesta/docentes/encuesta2.png}
    \caption{Aspectos útiles del sistema}
    \label{fig:p2_docente_utilidad}
\end{figure}

Los resultados de la encuesta indican que la "Facilidad del llenado del formulario del sílabo" fue la característica más valorada con un 68.8\% de respuestas positivas, seguida por la "Mejor organización y control de sus datos" con un 56.3\%. Tanto la "Generación automática del contenido del sílabo con IA" como el "Llenado automático del Plan de clase y la Guía de estudio" recibieron una valoración del 43.8\%. Finalmente, un 6.3\% de los encuestados mencionaron "Otras cualidades".

\paragraph{3. ¿Cómo evaluaría la facilidad de uso del sistema?}

\begin{table}[H]
    \centering
    \caption{Facilidad de uso del sistema}
    \label{tab:p3_docente_facilidad}
    \begin{tabular}{lc}
        \toprule
        \textbf{Facilidad de Uso PLANEAUML} & \textbf{Porcentaje} \\
        \midrule
        Muy Fácil & 50\% \\
        Fácil & 50\% \\
        Algo Complejo & 0\% \\
        Muy Difícil & 0\% \\
        \bottomrule
    \end{tabular}
\end{table}

\begin{figure}[H]
    \centering
    \includegraphics[width=0.8\textwidth]{figuras/segunda_encuesta/docentes/encuesta3.png}
    \caption{Facilidad de uso percibida}
    \label{fig:p3_docente_facilidad}
\end{figure}

El 100\% de los encuestados percibe el sistema como fácil de usar, dividiéndose equitativamente entre "Muy fácil" y "Fácil".

\paragraph{4. ¿Cuánto tiempo promedio dedica a elaborar un plan de estudios utilizando PLANEAUML?}

\begin{table}[H]
    \centering
    \caption{Tiempo promedio dedicado con PLANEAUML}
    \label{tab:p4_docente_tiempo}
    \begin{tabular}{lc}
        \toprule
        \textbf{Tiempo} & \textbf{Porcentaje} \\
        \midrule
        Menos de 1 hora & 81\% \\
        2 Horas & 6\% \\
        Más de 3 horas & 13\% \\
        \bottomrule
    \end{tabular}
\end{table}

\begin{figure}[H]
    \centering
    \includegraphics[width=0.8\textwidth]{figuras/segunda_encuesta/docentes/encuesta4.png}
    \caption{Tiempo de elaboración de planes}
    \label{fig:p4_docente_tiempo}
\end{figure}

El 81\% de los usuarios elabora un plan de estudios en menos de una hora utilizando PLANEAUML. Previamente, con métodos tradicionales, el 70\% de los docentes tardaba entre 2 y 5 horas. Esta reducción de tiempo se atribuye a mecanismos del sistema como la automatización de tareas repetitivas, la generación automática de sílabos con IA que precompleta secciones, y la validación de datos en tiempo real que alerta sobre errores o campos incompletos.

\paragraph{5. ¿Considera que el sistema PLANEAUML reduce los tiempos de planeación comparada con la forma tradicional?}

\begin{table}[H]
    \centering
    \caption{Reducción de tiempos de planeación}
    \label{tab:p5_docente_reduccion}
    \begin{tabular}{lc}
        \toprule
        \textbf{Reducción de Tiempo} & \textbf{Porcentaje} \\
        \midrule
        Sí & 100\% \\
        No & 0\% \\
        \bottomrule
    \end{tabular}
\end{table}

\begin{figure}[H]
    \centering
    \includegraphics[width=0.8\textwidth]{figuras/segunda_encuesta/docentes/encuesta5.png}
    \caption{Percepción de reducción de tiempo}
    \label{fig:p5_docente_reduccion}
\end{figure}

La totalidad de los encuestados reconoce que el sistema para la planificación docente PLANEAUML reduce los tiempos de planificación en comparación con los métodos tradicionales.

\paragraph{6. ¿Considera que el sistema reduce los errores en la elaboración de sílabos y planes de estudio?}

\begin{table}[H]
    \centering
    \caption{Reducción de errores en la elaboración}
    \label{tab:p6_docente_errores}
    \begin{tabular}{lc}
        \toprule
        \textbf{Reducción de errores} & \textbf{Porcentaje} \\
        \midrule
        Sí & 94\% \\
        No & 6\% \\
        \bottomrule
    \end{tabular}
\end{table}

\begin{figure}[H]
    \centering
    \includegraphics[width=0.8\textwidth]{figuras/segunda_encuesta/docentes/encuesta6.png}
    \caption{Reducción de errores}
    \label{fig:p6_docente_errores}
\end{figure}

Una amplia mayoría (94\%) percibe que PLANEAUML contribuye a reducir errores en la elaboración de sílabos y planes de estudio.

\paragraph{7. ¿Cómo calificaría la generación automática de sílabos mediante el uso de inteligencia artificial?}

\begin{table}[H]
    \centering
    \caption{Evaluación de la generación automática con IA}
    \label{tab:p7_docente_ia}
    \begin{tabular}{lc}
        \toprule
        \textbf{Evaluación de la eficiencia de la IA} & \textbf{Porcentaje} \\
        \midrule
        Muy Buena & 69\% \\
        Buena & 31\% \\
        Regular & 0\% \\
        Mala & 0\% \\
        \bottomrule
    \end{tabular}
\end{table}

\begin{figure}[H]
    \centering
    \includegraphics[width=0.8\textwidth]{figuras/segunda_encuesta/docentes/encuesta7.png}
    \caption{Calificación de la IA}
    \label{fig:p7_docente_ia}
\end{figure}

El 100\% de los encuestados califica positivamente la funcionalidad de generación automática de sílabos mediante inteligencia artificial, con un 69\% considerándola "Muy buena" y un 31\% "Buena".

\paragraph{8. ¿Cree que PLANEAUML puede contribuir a mejorar la planificación de los docentes?}

\begin{table}[H]
    \centering
    \caption{Contribución a la mejora de la planificación}
    \label{tab:p8_docente_mejora}
    \begin{tabular}{lc}
        \toprule
        \textbf{Aceptación de PLANEAUML} & \textbf{Porcentaje} \\
        \midrule
        Sí & 100\% \\
        Tal vez & 0\% \\
        No & 0\% \\
        \bottomrule
    \end{tabular}
\end{table}

\begin{figure}[H]
    \centering
    \includegraphics[width=0.8\textwidth]{figuras/segunda_encuesta/docentes/encuesta8.png}
    \caption{Mejora en la planificación}
    \label{fig:p8_docente_mejora}
\end{figure}

Todos los encuestados coincidieron en que PLANEAUML puede mejorar la planificación docente. El sistema automatiza tareas como el llenado de formularios y la organización de datos, y los usuarios perciben su facilidad de uso.

\subsection{Tercera encuesta aplicada seis meses después de la implementación de PLANEAUML}

\subsubsection{Introducción}
El presente capítulo detalla los hallazgos obtenidos tras la aplicación del instrumento de recolección de datos a los docentes usuarios del sistema PLANEAUML. La investigación se centró en evaluar la experiencia de usuario, el nivel de integración tecnológica, el impacto en la eficiencia administrativa y la percepción sobre la calidad educativa tras seis meses de implementación del sistema.

Los resultados se presentan organizados en cuatro dimensiones clave: (1) Nivel de adopción e integración en la rutina docente, (2) Transformación de la experiencia de planificación (Antes vs. Después), (3) Repercusión en la labor pedagógica y (4) El rol de la Inteligencia Artificial en el proceso.

\subsubsection{Nivel de Adopción e Integración del Sistema}
Uno de los indicadores principales para medir el éxito de la implementación tecnológica es el grado en que los usuarios incorporan la herramienta en su flujo de trabajo diario. Al consultar a los docentes sobre qué tan integrada sienten la plataforma en su rutina (en una escala del 1 al 10), los resultados mostraron una aceptación contundente.

\begin{figure}[H]
    \centering
    \includegraphics[width=0.8\textwidth]{figuras/tercera_encuesta/encuesta1.png}
    \caption{Gráfico de barras sobre nivel de integración}
    \label{fig:tercera_encuesta_figura1}
\end{figure}

Como se observa en la Figura \ref{fig:tercera_encuesta_figura1}, el proceso de adopción de PLANEAUML ha sido exitoso y homogéneo. El 100\% de las respuestas se situaron en el rango superior (8 a 10). Específicamente, el promedio general de integración es de 8.9, donde ningún docente calificó la herramienta por debajo de 8. Los participantes destacan que factores como la "centralización de la información" y la "accesibilidad desde cualquier lugar" han sido determinantes para este alto nivel de adhesión.

\subsubsection{Transformación de la Rutina de Trabajo: Eficiencia y Organización}
El estudio comparativo entre los métodos tradicionales y el nuevo sistema digital revela un cambio de paradigma significativo. Los docentes describieron su experiencia previa (basada en archivos dispersos de Word y Excel) como procesos "estáticos", "repetitivos" y propensos a la desorganización.

La implementación de PLANEAUML ha transformado esta dinámica hacia un modelo centralizado. El análisis de las respuestas abiertas permite visualizar este contraste de manera clara:

\begin{figure}[H]
    \centering
    \includegraphics[width=0.8\textwidth]{figuras/tercera_encuesta/encuesta4.png}
    \caption{Diagrama comparativo (Antes vs. Después)}
    \label{fig:tercera_encuesta_figura4}
\end{figure}

Tal como se resume en la Figura \ref{fig:tercera_encuesta_figura4}, los docentes han transicionado de una percepción de "desastre" o desorden documental a una de "control de datos". Los hallazgos cualitativos indican que el beneficio más valorado no es solo la velocidad, sino la organización.

Un participante señaló: "Anteriormente, la información académica y pedagógica se encontraba dispersa. Con PLANEAUML, todo se ha centralizado". Otro docente reforzó esta idea indicando que "el sistema reduce significativamente la necesidad de manejar múltiples archivos". Asimismo, las funciones de exportación a formatos estándar (Word/Excel) fueron calificadas como "muy útiles" para la generación de reportes institucionales, eliminando la necesidad de redigitar información.

\subsubsection{Impacto de la Eficiencia Administrativa en la Calidad Educativa}
Una de las hipótesis centrales de la digitalización es que, al reducir la carga burocrática, el docente puede dedicar más tiempo a actividades de valor pedagógico. Los datos recabados confirman esta premisa. Al preguntar a los docentes si el ahorro de tiempo les ha permitido enfocarse en otros aspectos de su docencia, la respuesta fue unánimemente positiva.

\begin{figure}[H]
    \centering
    \includegraphics[width=0.8\textwidth]{figuras/tercera_encuesta/encuesta2.png}
    \caption{Gráfico de anillo (Reinversión del tiempo)}
    \label{fig:tercera_encuesta_figura2}
\end{figure}

La Figura \ref{fig:tercera_encuesta_figura2} ilustra el destino de este "tiempo recuperado". El 50\% de las menciones se refieren a la mejora en la preparación de clases y búsqueda de recursos didácticos más dinámicos. Un 30\% de los docentes indicó que ahora pueden ofrecer una retroalimentación más detallada y personalizada a los estudiantes, algo que antes se dificultaba por la carga administrativa. El 20\% restante aprovecha este tiempo para la investigación propia.

Estos resultados sugieren que PLANEAUML no solo funciona como una herramienta administrativa, sino que actúa como un catalizador para la mejora de la calidad educativa, permitiendo al docente centrarse en el proceso de enseñanza-aprendizaje en lugar del llenado de formatos.

\subsubsection{El Rol de la Inteligencia Artificial en la Planificación}
La incorporación de Inteligencia Artificial (IA) generativa para la creación de sílabos representa la característica más innovadora del sistema. Sin embargo, es crucial entender cómo interactúan los docentes con esta tecnología. Los resultados desmitifican la idea de que la IA reemplaza la labor intelectual del docente; por el contrario, actúa como un asistente.

\begin{figure}[H]
    \centering
    \includegraphics[width=0.8\textwidth]{figuras/tercera_encuesta/encuesta3.png}
    \caption{Gráfico de pastel (Frecuencia de uso de IA)}
    \label{fig:tercera_encuesta_figura3}
\end{figure}

Como muestra la Figura \ref{fig:tercera_encuesta_figura3}, existe un uso estratégico y moderado de la herramienta. La mayoría de los docentes (sumando las categorías "Ocasional" y "Regular") utiliza la IA como un mecanismo de apoyo.

El análisis cualitativo de las respuestas define tres funciones principales que los docentes asignan a la IA:
\begin{itemize}
    \item \textbf{Punto de Partida:} Para vencer el "síndrome de la hoja en blanco" y generar estructuras iniciales.
    \item \textbf{Generación de Ideas:} Para encontrar nuevas estrategias didácticas o redacción de competencias.
    \item \textbf{Optimización de Tiempo:} Para redactar secciones formales rápidamente.
\end{itemize}

Es importante destacar que los docentes mantienen un rol de supervisión crítica. Un testimonio relevante menciona: "La IA ayuda en ideas nada más, porque para redactar las secciones a veces lo hace de manera muy larga y poco pedagógica". Esto evidencia que el sistema requiere y fomenta la revisión humana, validando la insustituibilidad del criterio docente.

\subsubsection{Estandarización Institucional y Coherencia}
Más allá de la experiencia individual, los resultados apuntan a una mejora a nivel institucional. Los docentes coinciden en que el uso de una plataforma estandarizada ha mejorado la "coherencia" de los planes de estudio en toda la universidad. Al seguir un mismo formato y estructura lógica impuesta por el software, se ha eliminado la disparidad que existía cuando cada docente gestionaba sus propios archivos. Esto facilita a la coordinación académica el seguimiento y la evaluación de las asignaturas.

\subsubsection{Percepción de Mejoras y Recomendaciones de los Usuarios}
A pesar de la alta satisfacción, el estudio recogió áreas de oportunidad identificadas por los usuarios, lo que demuestra un uso crítico y profundo de la herramienta. Las principales solicitudes de mejora incluyen:
\begin{itemize}
    \item \textbf{Integración de Documentos:} Varios docentes sugirieron unificar la "Guía de Estudio Independiente" con el "Plan de Clase" para evitar la duplicidad de entradas.
    \item \textbf{Notificaciones:} Se solicitó la implementación de alertas automáticas o recordatorios de fechas de entrega.
    \item \textbf{Capacitación Continua:} Petición de tutoriales integrados para funciones avanzadas.
\end{itemize}

Finalmente, al proyectar el futuro de la digitalización en la universidad, los docentes recomendaron expandir este modelo hacia otros procesos, específicamente la gestión de calificaciones, el control de asistencia y el seguimiento al desempeño estudiantil, sugiriendo una integración total con el Sistema de Administración Universitaria (SAU).
