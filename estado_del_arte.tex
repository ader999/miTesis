% Estado del Arte
% Este archivo contiene la revisión del estado del arte de la investigación

El presente capítulo aborda el estado actual del conocimiento y la tecnología en el ámbito de la gestión académica universitaria, con el fin de contextualizar y fundamentar la necesidad, pertinencia y originalidad del sistema web PLANEAUML. La transición hacia la digitalización en la educación superior ha impulsado el desarrollo de múltiples herramientas, sin embargo, persisten brechas específicas en la optimización de la planificación docente, área central de esta investigación.


\subsection{El Panorama de los Sistemas de Gestión Universitaria}

La gestión universitaria moderna se apoya cada vez más en soluciones de software robustas diseñadas para centralizar y automatizar una amplia gama de procesos administrativos y académicos. Estas herramientas, a menudo clasificadas como Sistemas de Información Estudiantil (SIS) o Planificación de Recursos Empresariales (ERP) educativos, buscan integrar funciones como la gestión de admisiones, matrículas, registros financieros, horarios y comunicación institucional. [1]Plataformas como Odoo, Fedena y OpenEduCat son ejemplos de soluciones que ofrecen módulos para cubrir estas necesidades, permitiendo a las instituciones unificar su operación.

Sin embargo, un análisis de estas plataformas comerciales revela que, si bien son exhaustivas en la gestión administrativa general, a menudo tratan la planificación curricular y la elaboración de sílabos como un componente secundario. Su enfoque principal es la gestión de datos a nivel macro (estudiantes, cursos, finanzas), prestando menos atención a las herramientas pedagógicas específicas que los docentes necesitan para diseñar y detallar sus asignaturas. Como señala la guía de software de ClickUp, las funcionalidades clave buscadas en estas plataformas incluyen la gestión de inscripciones, información de estudiantes y la automatización de flujos de trabajo administrativos, pero no se profundiza en la creación de contenido académico en sí. [1]Esta característica generalista deja un vacío para herramientas más especializadas que se centren en el núcleo del proceso de enseñanza-aprendizaje.

\subsection{Digitalización de Planes de Estudio}

