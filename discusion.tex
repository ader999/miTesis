% !TEX root = main.tex
\label{sec:discusion}

La presente investigación tuvo como objetivo general evaluar el impacto de la implementación del sistema web PLANEAUML en la optimización de la gestión de planes de estudio y la elaboración de sílabos en la Universidad Martín Lutero, Sede Jalapa. Tras el análisis de los datos recabados mediante un diseño mixto de triangulación concurrente, los hallazgos confirman la hipótesis de trabajo: la intervención tecnológica no solo optimizó la eficiencia operativa, sino que transformó positivamente la experiencia docente y la cultura organizacional de la institución.

A continuación, se discuten estos resultados a la luz de los antecedentes, el marco teórico y las preguntas de investigación planteadas.

\subsection{Eficiencia Operativa: Del Proceso Manual a la Automatización}

La primera pregunta específica de esta investigación indagaba sobre la comparación de la eficiencia operativa entre los métodos tradicionales y el nuevo sistema. Los resultados cuantitativos son contundentes al demostrar una mejora sustancial. Antes de la implementación, el 70\% de los docentes dedicaba entre 2 y 5 horas a la elaboración de un plan de estudio, utilizando herramientas genéricas como Word y Excel que, si bien son funcionales, no están diseñadas para la gestión curricular estructurada. Con el uso de PLANEAUML, el 81\% de los usuarios logró completar la misma tarea en menos de una hora.

Este hallazgo es consistente con lo reportado por \cite{silva2022sistema}, quien encontró que la implementación de sistemas web responsivos reduce significativamente la desorganización y agiliza los procesos administrativos. Sin embargo, el impacto observado en este estudio supera las métricas de simple digitalización. La reducción del tiempo en más de un 50\% para la mayoría de la muestra sugiere que la integración de bases de datos relacionales (PostgreSQL) y la automatización de formatos predefinidos eliminan eficazmente los "cuellos de botella" del método manual, como el copiado y pegado repetitivo de información institucional o la reestructuración de formatos desconfigurados.

Asimismo, la reducción de la frecuencia de errores y retrabajos, validada por el 94\% de los docentes en la fase post-implementación, contrasta con el 40\% que reportaba problemas frecuentes en el diagnóstico inicial. Esto corrobora la utilidad de las validaciones lógicas implementadas en el sistema (Django), las cuales actúan como filtros preventivos que aseguran la integridad de los datos antes de que estos se conviertan en documentos finales.

\subsection{La Inteligencia Artificial como Asistente Cognitivo}

Uno de los aportes más novedosos de esta investigación fue la evaluación de la inteligencia artificial generativa aplicada a la planificación curricular. Contrario a las preocupaciones éticas o laborales que a menudo surgen con la adopción de IA en la educación, los resultados de este estudio indican una integración armónica y complementaria.

El 100\% de los docentes calificó la funcionalidad de IA como "Buena" o "Muy Buena". Sin embargo, el análisis cualitativo revela un matiz crucial: la IA no reemplazó al docente, sino que lo asistió. Las categorías emergentes del análisis fenomenológico muestran que los profesores utilizan la herramienta principalmente para superar el "bloqueo de la hoja en blanco" y para generar ideas iniciales, manteniendo siempre el control editorial final.

Este comportamiento valida la postura de que la tecnología en la educación debe servir como un andamiaje cognitivo. A diferencia de los antecedentes revisados, como los de \cite{cotillo2017implementacion} o \cite{mora2018sistema}, que se centraron en la gestión administrativa del dato, PLANEAUML da un paso adelante al asistir en la \textit{creación} del contenido pedagógico. Esto demuestra que es posible integrar modelos de lenguaje avanzados en procesos burocráticos para aportar valor creativo y no solo logístico.

\subsection{Triangulación de Resultados: La Reinversión del Tiempo}

La triangulación de los datos cuantitativos y cualitativos permite una comprensión más profunda del impacto del sistema. Si bien el dato cuantitativo destaca la "reducción de tiempo", el dato cualitativo explica el "valor" de ese ahorro.

Se identificó un fenómeno de "reinversión pedagógica". El tiempo liberado por la automatización no se tradujo simplemente en menos horas de trabajo, sino en un cambio en la naturaleza del trabajo docente. Los participantes reportaron dedicar ese tiempo extra a la preparación de clases más dinámicas y a brindar retroalimentación más detallada a los estudiantes. Esto es significativo porque sugiere que la ineficiencia administrativa previa estaba canibalizando tiempo valioso de la enseñanza.

Al eliminar la carga mecánica de la planificación, PLANEAUML permitió a los docentes reconectarse con su función principal: la mediación pedagógica. Esto se alinea con el paradigma pragmático de la investigación, donde la utilidad de la herramienta se mide por su capacidad para mejorar la realidad educativa práctica. La satisfacción del 100\% reportada en la encuesta final no es solo por la facilidad de uso del software, sino por la percepción de bienestar y profesionalización que este genera.

\subsection{Implicaciones Institucionales y Estandarización}

Desde una perspectiva organizacional, la investigación evidencia que la tecnología actúa como un agente normalizador. Antes de la intervención, la diversidad de formatos y estilos dificultaba el seguimiento académico. La adopción de PLANEAUML impuso, de manera implícita pero efectiva, una estandarización de la estructura curricular en la Sede Jalapa.

Esto tiene implicaciones directas para la gestión de la calidad. Al centralizar la información y unificar criterios, la universidad gana en capacidad de monitoreo y evaluación, tal como lo sugería el Plan Estratégico Institucional \cite{unan2020plan} al priorizar la modernización de la gestión. La disponibilidad de datos estructurados abre la puerta a futuros análisis de analítica de aprendizaje (Learning Analytics) que no eran posibles con los archivos dispersos de Word y Excel.

\subsection{Limitaciones y Líneas Futuras}

Es importante reconocer las limitaciones de estos hallazgos. El estudio se circunscribió a una única sede universitaria y a una muestra relativamente pequeña, lo que limita la generalización estadística de los resultados a contextos con infraestructuras o culturas organizacionales muy diferentes. Además, la evaluación se centró en la perspectiva del docente y el administrativo, sin medir directamente el impacto en el rendimiento académico de los estudiantes.

No obstante, la consistencia interna de los resultados (triangulación) otorga validez al modelo propuesto. Las líneas futuras de investigación y las recomendaciones prácticas derivadas de estos hallazgos se detallan en la siguiente sección.

En conclusión, la discusión de los resultados permite afirmar que PLANEAUML ha cumplido con su propósito de optimización, estableciendo un nuevo estándar de gestión académica en la Universidad Martín Lutero, Sede Jalapa, y demostrando que la transformación digital, cuando está centrada en el usuario, es un motor potente para la mejora educativa.